\documentclass[12pt]{book}
\usepackage[spanish]{babel}

%%%%%%%%%%%%%%%%%%%%%%%%%%%%%%%%%%
%%%%%%%%%%%%%%%%%%%%%%%%%%%%%   %%
%%        Datos Trabajo     %%  %%
%%%%%%%%%%%%%%%%%%%%%%%%%%%%%%%%%%
\newcommand{\titulo}[0]{Notas de clase}
\newcommand{\materia}[0]{Inteligencia Artificial y Teor\'ia de la Computaci\'on}
\newcommand{\grupo}[0]{\'Unico}
\newcommand{\unidad}[0]{Seccion}

\newcommand{\periodo}[0]{Ago-Dic 2020}


\newcommand{\universidad}[0]{Universidad de Guanajuato}
\newcommand{\curso}[0]{Lic. en Computaci\'on Matem\'atica}


%%%%%%%%%%%%%%%%%%%%%%%%%%%%%%%%%%
%%%%%%%%%%%%%%%%%%%%%%%%%%%%%%%%%%
\usepackage{amssymb}
\usepackage{amsthm}
\usepackage{enumerate}
\usepackage{mathtools}
\usepackage{multicol}
\usepackage{soul}
\usepackage{thmtools}
	
\usepackage{geometry}
	\geometry{}

\usepackage{graphicx}
	\graphicspath{ {assets/} }

\usepackage{hyperref}
	\hypersetup{
			pdftex,
		        pdfauthor={bench},
		        pdftitle={\titulo},
		        pdfsubject={\materia},
		        pdfkeywords={\grupo, \unidad, UG},
		        pdfproducer={Latex with hyperref, Ubuntu},
		        pdfcreator={pdflatex, gummi},
			colorlinks=true,
				linkcolor=blue,
				urlcolor=cyan,
				filecolor=yellow}
%%%%%%%%%%%%%%%%%%%%%%%%%%%%%%%%%%
%%%%%%%%%%%%%%%%%%%%%%%%%%%%%%%%%%		
		
\newtheorem{teorema}{Teorema}[section]
\newtheorem{lemma}{Lemma}[teorema]

\newtheorem{definicion}{Definicion}[part]

%%%%%%%%%%%%%%%%%%%%%%%%%%%%%%%%%%
%%%%%%%%%%%%%%%%%%%%%%%%%%%%%%%%%%

\title{
	\ \\ Benjam\'in Rivera \\
	\bf{\titulo}\\\ \\}

\author{
	\universidad \\
	\textit{\curso} \\
	\textit{Materia:} \materia \\
	\textit{Grupo:} \grupo \\
	\textit{Periodo:} \periodo }

\date{\textit{\'Ultima actualizaci\'on:} \today}


%%%%%%%%%%%%%%%%%%%%%%%%%%%%%
%%        Documento         %%
%%%%%%%%%%%%%%%%%%%%%%%%%%%%%%%
\begin{document}
\maketitle

\section*{Informaci\'on general}

\par Notas del curso \textbf{\materia} impartido por los profesores \textit{Jes\'us Rodr\'iguez Viorato} (jesus@cimat.mx) y \textit{Arturo Hern\'andez Aguirre} (artha@cimat.mx). Este curso fue ofrecido para los estudiantes de la \textit{\curso} durante el periodo \textit{\periodo}.

\par Dado el panorama del momento, las clases seran ofrecidas en modalidad a distancia. En el caso de este curso se decidio por usar \textit{Classroom} y \textit{BlueJueans} para organizar el curso y realizar videoconferencias, respectivamente. Es importante tener en cuenta que, a pesar de que el curso es virtual, no es autogestivo; esto implica que se debe asistir a clases en un horario especifico.

\subsubsection*{Calificaci\'on}

\par La calificaci\'on de este curso estara dividida en dos partes
	\begin{quote} \begin{enumerate}[I]
		\item \textbf{Teor\'ia de la computaci\'on} $50\%$
			\begin{enumerate}
				\item Tareas $0\%$
				\item Participaci\'on $30\%$
				\item Ex\'amenes $70\%$
			\end{enumerate}		
		\item \textbf{Inteligencia Artificial} $50\%$
			\begin{enumerate}
				\item Pendiente
				\item ...
			\end{enumerate}
	\end{enumerate} \end{quote}



\listoftheorems
\tableofcontents


\part{Teor\'ia de la Computaci\'on}

\section*{Biblograf\'ia}
	
	\par Sugeridas
	\begin{enumerate}
		\item Introduction to Languages and the Theory of Computation. Jhon C. Martin
		\item Sipser, Michael. Introduction to the Theory of Computation. Thomson Course Technology, 2006.
		\item Hopcroft, John E. et. Al. Introduction to Automata Theory, Languages and Computation. Pearson. 2008.
	\end{enumerate}
	
	\par Extras
	\begin{enumerate}
		\item Viso G., E. (2015). \textit{Introducción a Autómatas y Lenguajes Formales} (2.a ed.). Las prensas de Ciencia.
	\end{enumerate}

\chapter{Aut\'omatas}

	\par La historia corta es que los \textit{Aut\'omatas} son  \textbf{maquinas de estados} que usamos para representar procesos. No todos los aut\'omatas tienen las mismas capacidades, y despu\'es estos son usados para calcular la complejidad de los procesos que pueden, o no, ser representados. Despu\'es de trabajar un rato con ellos llegamos a la instancia m\'as importante de estos, la \textbf{M\'aquina de Turing}; que como dato curioso podemos decir que nuestras computadoras actuales (las no cu\'anticas) son intentos de reporoducir a este aut\'omata.
	\begin{quote}
		La teor\'ia de aut\'omatas es el modelo computacional m\'as sencillo.
	\end{quote}
	Posteriormente se usara a estos para estudiar conceptos como \textbf{computabilidad} y \textbf{complejidad}.
	
	\par Todo aut\'omata esta compuesto de cuatro partes escenciales; \textit{estados, estados iniciales, estados aceptores y relaciones de transici\'on}; en la figura \ref{fig: automata}. Despu\'es se van agregando partes y limitantes, pero estas son las partes escenciales de estos.

	\begin{figure}[h]
		\centering
			\includegraphics[scale=0.5]{example-image}
		\caption{Aut\'omata Finito}
		\label{fig: automata}
	\end{figure}
	
	\par Todos los aut\'omatas trabajaran con cadenas pertenecientes al \textit{alfabeto*} y, siguiendo las relaciones de transici\'on, diremos que una \textit{cadena es aceptada} si y solo si el estado final al que se llega con dicha cadena esta contenida entre los \textit{estados aceptores}.
	
	\par Adem\'as, todos los aut\'omatas $A$ se pueden asociar a un lenguaje $L$. De manera que, si para cualquier cadena $l \in L$ es cierto que la cadena $l$ es aceptada por el au\'omata $A$, entonces decimos que $L$ es regular.
	
	
	\section{Aut\'omatas Finitos}
	
		\par Los \textit{Aut\'omatas Finitos} \textbf{(AF)} son los m\'as simples de los aut\'omatas que se estudiaran en este curso. Su definici\'on formal es
		
		\begin{definicion}[AF]
			Un \textbf{Aut\'omata Finito (AF)} es una $5\text{-tupla}$ $(Q, \Sigma, \delta, q_0, F)$ donde
			\begin{enumerate}
				\item $Q$ es un conjunto finito que representa a los \textbf{estados}
				\item $\Sigma$ es un conjunto finito que representa el \textbf{alfabeto}
				\item $\delta$ es una aplicaci\'on tal que $\delta:Q\times\Sigma \mapsto Q$ como la funci\'on de transici\'on.
				\item $q_0 \in Q$ como el estado inicial.
				\item $F\subseteq Q$ como el conjunto de los estados aceptores. 
			\end{enumerate}
		\end{definicion}

		\subsection{Cadenas y Lenguajes}
		
		\par Como ya se hab\'ia mencionado, una cadena $c$ es un arreglo finito de elementos pertenecientes a un alfabeto $\Sigma$.
		
		\begin{definicion}[$\Sigma^*$]
			Se define a $\Sigma^*$ como al conjunto de todas las posibles combinaciones finitas de los elementos de $\Sigma$
		\end{definicion}
		
		\par Luego, a los subconjuntos de $\Sigma^*$ que cumplan con alguna caracteristica en especifico \footnote{como ser aceptados por un aut\'omata xP} los llamamos \textbf{Lenguajes}. Entre cadenas y lenguajes una de las operaciones m\'as comunes es la \textbf{concatenaci\'on}
		
		\begin{definicion}[Concatenaci\'on lenguajes]
			Sean $L_1, L_2$ lenguajes dados. La concatenaci\'on entre lenguajes $L_1L_2$ queda definida por
			$$ L_1\circ L_2 = \{ w_1, w_2 \in \Sigma^* | w_1\in L_1 \wedge w_2\in L_2\}$$
		\end{definicion}
		
		En la definici\'on anterior existen dos formas aceptadas para denotar la concatenaci\'on de dos lenguajes; de manera que $L_1\circ L_2$ es lo mismo que $L_1 L_2$. Adem\'as, como con la uni\'on de lenguajes regulares, se da que
		
		\begin{teorema}
			Si $L_1, L_2$ son lenguajes regulares, entonces existe un aut\'omata tal que la concatenaci\'on $L_1\circ L_2$ es regular.
		\end{teorema}
	
		La prueba de esta secci\'on se queda pendiente para la Secci\'on~\ref{sec: AFND}



	\section{Aut\'omatas Finitos No Deterministas} \label{sec: AFND}
	
	\par Los \textbf{Aut\'omatas Finitos No Deterministas (AFND)} se pueden considerar como una ampliaci\'on de los \textbf{AF}\footnote{Aunque m\'as adelante se demostrara que estos son equivalentes}. Estos permiten bosquejar aut\'omatas m\'as rapido e intuitivos que antes.
	
	\par Las nuevas reglas que los AFND introducen, en comparaci\'on con los AF, son las siguientes:
	\begin{enumerate}
		\item Permite repetir transiciones de un nodo.
		\item Puede haber nodos que no otorguen una transici\'on para cada caracter de $\Sigma$.
		\item Se introducen las transiciones vacias.
	\end{enumerate}
	
	\begin{figure}[htp]
		\centering
			\includegraphics[scale=0.4]{example-image}
		\caption{Aut\'omata Finifto No Determinista}
		\label{fig: AFND}
	\end{figure}
	
	\par Lo primero que se debe notar como diferencia es que, para una cadena, existiran distitnos estados finales (y algunos de los caminos pueden \textit{romperse} antes de leer toda la cadena). 


% section AFND (end)
	
	






\chapter{Computabilidad}
	\begin{quote}
		La \textbf{computabilidad} implica resolver la pregunta \textit{Se puede resolver?}
	\end{quote}
	
\chapter{Complejidad}
	\begin{quote}
		La \textbf{complejidad} trata de encontrar una medida respecto a que tan complicado es un problema.
	\end{quote}


\part{Inteligencia Artificial}


\end{document}