\documentclass[11pt]{article}

    \usepackage[breakable]{tcolorbox}
    \usepackage{parskip} % Stop auto-indenting (to mimic markdown behaviour)
    
    \usepackage{iftex}
    \ifPDFTeX
    	\usepackage[T1]{fontenc}
    	\usepackage{mathpazo}
    \else
    	\usepackage{fontspec}
    \fi

    % Basic figure setup, for now with no caption control since it's done
    % automatically by Pandoc (which extracts ![](path) syntax from Markdown).
    \usepackage{graphicx}
    % Maintain compatibility with old templates. Remove in nbconvert 6.0
    \let\Oldincludegraphics\includegraphics
    % Ensure that by default, figures have no caption (until we provide a
    % proper Figure object with a Caption API and a way to capture that
    % in the conversion process - todo).
    \usepackage{caption}
    \DeclareCaptionFormat{nocaption}{}
    \captionsetup{format=nocaption,aboveskip=0pt,belowskip=0pt}

    \usepackage[Export]{adjustbox} % Used to constrain images to a maximum size
    \adjustboxset{max size={0.9\linewidth}{0.9\paperheight}}
    \usepackage{float}
    \floatplacement{figure}{H} % forces figures to be placed at the correct location
    \usepackage{xcolor} % Allow colors to be defined
    \usepackage{enumerate} % Needed for markdown enumerations to work
    \usepackage{geometry} % Used to adjust the document margins
    \usepackage{amsmath} % Equations
    \usepackage{amssymb} % Equations
    \usepackage{textcomp} % defines textquotesingle
    % Hack from http://tex.stackexchange.com/a/47451/13684:
    \AtBeginDocument{%
        \def\PYZsq{\textquotesingle}% Upright quotes in Pygmentized code
    }
    \usepackage{upquote} % Upright quotes for verbatim code
    \usepackage{eurosym} % defines \euro
    \usepackage[mathletters]{ucs} % Extended unicode (utf-8) support
    \usepackage{fancyvrb} % verbatim replacement that allows latex
    \usepackage{grffile} % extends the file name processing of package graphics 
                         % to support a larger range
    \makeatletter % fix for grffile with XeLaTeX
    \def\Gread@@xetex#1{%
      \IfFileExists{"\Gin@base".bb}%
      {\Gread@eps{\Gin@base.bb}}%
      {\Gread@@xetex@aux#1}%
    }
    \makeatother

    % The hyperref package gives us a pdf with properly built
    % internal navigation ('pdf bookmarks' for the table of contents,
    % internal cross-reference links, web links for URLs, etc.)
    \usepackage{hyperref}
    % The default LaTeX title has an obnoxious amount of whitespace. By default,
    % titling removes some of it. It also provides customization options.
    \usepackage{titling}
    \usepackage{longtable} % longtable support required by pandoc >1.10
    \usepackage{booktabs}  % table support for pandoc > 1.12.2
    \usepackage[inline]{enumitem} % IRkernel/repr support (it uses the enumerate* environment)
    \usepackage[normalem]{ulem} % ulem is needed to support strikethroughs (\sout)
                                % normalem makes italics be italics, not underlines
    \usepackage{mathrsfs}
    

    
    % Colors for the hyperref package
    \definecolor{urlcolor}{rgb}{0,.145,.698}
    \definecolor{linkcolor}{rgb}{.71,0.21,0.01}
    \definecolor{citecolor}{rgb}{.12,.54,.11}

    % ANSI colors
    \definecolor{ansi-black}{HTML}{3E424D}
    \definecolor{ansi-black-intense}{HTML}{282C36}
    \definecolor{ansi-red}{HTML}{E75C58}
    \definecolor{ansi-red-intense}{HTML}{B22B31}
    \definecolor{ansi-green}{HTML}{00A250}
    \definecolor{ansi-green-intense}{HTML}{007427}
    \definecolor{ansi-yellow}{HTML}{DDB62B}
    \definecolor{ansi-yellow-intense}{HTML}{B27D12}
    \definecolor{ansi-blue}{HTML}{208FFB}
    \definecolor{ansi-blue-intense}{HTML}{0065CA}
    \definecolor{ansi-magenta}{HTML}{D160C4}
    \definecolor{ansi-magenta-intense}{HTML}{A03196}
    \definecolor{ansi-cyan}{HTML}{60C6C8}
    \definecolor{ansi-cyan-intense}{HTML}{258F8F}
    \definecolor{ansi-white}{HTML}{C5C1B4}
    \definecolor{ansi-white-intense}{HTML}{A1A6B2}
    \definecolor{ansi-default-inverse-fg}{HTML}{FFFFFF}
    \definecolor{ansi-default-inverse-bg}{HTML}{000000}

    % commands and environments needed by pandoc snippets
    % extracted from the output of `pandoc -s`
    \providecommand{\tightlist}{%
      \setlength{\itemsep}{0pt}\setlength{\parskip}{0pt}}
    \DefineVerbatimEnvironment{Highlighting}{Verbatim}{commandchars=\\\{\}}
    % Add ',fontsize=\small' for more characters per line
    \newenvironment{Shaded}{}{}
    \newcommand{\KeywordTok}[1]{\textcolor[rgb]{0.00,0.44,0.13}{\textbf{{#1}}}}
    \newcommand{\DataTypeTok}[1]{\textcolor[rgb]{0.56,0.13,0.00}{{#1}}}
    \newcommand{\DecValTok}[1]{\textcolor[rgb]{0.25,0.63,0.44}{{#1}}}
    \newcommand{\BaseNTok}[1]{\textcolor[rgb]{0.25,0.63,0.44}{{#1}}}
    \newcommand{\FloatTok}[1]{\textcolor[rgb]{0.25,0.63,0.44}{{#1}}}
    \newcommand{\CharTok}[1]{\textcolor[rgb]{0.25,0.44,0.63}{{#1}}}
    \newcommand{\StringTok}[1]{\textcolor[rgb]{0.25,0.44,0.63}{{#1}}}
    \newcommand{\CommentTok}[1]{\textcolor[rgb]{0.38,0.63,0.69}{\textit{{#1}}}}
    \newcommand{\OtherTok}[1]{\textcolor[rgb]{0.00,0.44,0.13}{{#1}}}
    \newcommand{\AlertTok}[1]{\textcolor[rgb]{1.00,0.00,0.00}{\textbf{{#1}}}}
    \newcommand{\FunctionTok}[1]{\textcolor[rgb]{0.02,0.16,0.49}{{#1}}}
    \newcommand{\RegionMarkerTok}[1]{{#1}}
    \newcommand{\ErrorTok}[1]{\textcolor[rgb]{1.00,0.00,0.00}{\textbf{{#1}}}}
    \newcommand{\NormalTok}[1]{{#1}}
    
    % Additional commands for more recent versions of Pandoc
    \newcommand{\ConstantTok}[1]{\textcolor[rgb]{0.53,0.00,0.00}{{#1}}}
    \newcommand{\SpecialCharTok}[1]{\textcolor[rgb]{0.25,0.44,0.63}{{#1}}}
    \newcommand{\VerbatimStringTok}[1]{\textcolor[rgb]{0.25,0.44,0.63}{{#1}}}
    \newcommand{\SpecialStringTok}[1]{\textcolor[rgb]{0.73,0.40,0.53}{{#1}}}
    \newcommand{\ImportTok}[1]{{#1}}
    \newcommand{\DocumentationTok}[1]{\textcolor[rgb]{0.73,0.13,0.13}{\textit{{#1}}}}
    \newcommand{\AnnotationTok}[1]{\textcolor[rgb]{0.38,0.63,0.69}{\textbf{\textit{{#1}}}}}
    \newcommand{\CommentVarTok}[1]{\textcolor[rgb]{0.38,0.63,0.69}{\textbf{\textit{{#1}}}}}
    \newcommand{\VariableTok}[1]{\textcolor[rgb]{0.10,0.09,0.49}{{#1}}}
    \newcommand{\ControlFlowTok}[1]{\textcolor[rgb]{0.00,0.44,0.13}{\textbf{{#1}}}}
    \newcommand{\OperatorTok}[1]{\textcolor[rgb]{0.40,0.40,0.40}{{#1}}}
    \newcommand{\BuiltInTok}[1]{{#1}}
    \newcommand{\ExtensionTok}[1]{{#1}}
    \newcommand{\PreprocessorTok}[1]{\textcolor[rgb]{0.74,0.48,0.00}{{#1}}}
    \newcommand{\AttributeTok}[1]{\textcolor[rgb]{0.49,0.56,0.16}{{#1}}}
    \newcommand{\InformationTok}[1]{\textcolor[rgb]{0.38,0.63,0.69}{\textbf{\textit{{#1}}}}}
    \newcommand{\WarningTok}[1]{\textcolor[rgb]{0.38,0.63,0.69}{\textbf{\textit{{#1}}}}}
    
    
    % Define a nice break command that doesn't care if a line doesn't already
    % exist.
    \def\br{\hspace*{\fill} \\* }
    % Math Jax compatibility definitions
    \def\gt{>}
    \def\lt{<}
    \let\Oldtex\TeX
    \let\Oldlatex\LaTeX
    \renewcommand{\TeX}{\textrm{\Oldtex}}
    \renewcommand{\LaTeX}{\textrm{\Oldlatex}}
    % Document parameters
    % Document title
    \title{Tarea5\\M\'etodos Num\'ericos}
    \author{Benjamin Rivera}
    
    
    
    
% Pygments definitions
\makeatletter
\def\PY@reset{\let\PY@it=\relax \let\PY@bf=\relax%
    \let\PY@ul=\relax \let\PY@tc=\relax%
    \let\PY@bc=\relax \let\PY@ff=\relax}
\def\PY@tok#1{\csname PY@tok@#1\endcsname}
\def\PY@toks#1+{\ifx\relax#1\empty\else%
    \PY@tok{#1}\expandafter\PY@toks\fi}
\def\PY@do#1{\PY@bc{\PY@tc{\PY@ul{%
    \PY@it{\PY@bf{\PY@ff{#1}}}}}}}
\def\PY#1#2{\PY@reset\PY@toks#1+\relax+\PY@do{#2}}

\expandafter\def\csname PY@tok@w\endcsname{\def\PY@tc##1{\textcolor[rgb]{0.73,0.73,0.73}{##1}}}
\expandafter\def\csname PY@tok@c\endcsname{\let\PY@it=\textit\def\PY@tc##1{\textcolor[rgb]{0.25,0.50,0.50}{##1}}}
\expandafter\def\csname PY@tok@cp\endcsname{\def\PY@tc##1{\textcolor[rgb]{0.74,0.48,0.00}{##1}}}
\expandafter\def\csname PY@tok@k\endcsname{\let\PY@bf=\textbf\def\PY@tc##1{\textcolor[rgb]{0.00,0.50,0.00}{##1}}}
\expandafter\def\csname PY@tok@kp\endcsname{\def\PY@tc##1{\textcolor[rgb]{0.00,0.50,0.00}{##1}}}
\expandafter\def\csname PY@tok@kt\endcsname{\def\PY@tc##1{\textcolor[rgb]{0.69,0.00,0.25}{##1}}}
\expandafter\def\csname PY@tok@o\endcsname{\def\PY@tc##1{\textcolor[rgb]{0.40,0.40,0.40}{##1}}}
\expandafter\def\csname PY@tok@ow\endcsname{\let\PY@bf=\textbf\def\PY@tc##1{\textcolor[rgb]{0.67,0.13,1.00}{##1}}}
\expandafter\def\csname PY@tok@nb\endcsname{\def\PY@tc##1{\textcolor[rgb]{0.00,0.50,0.00}{##1}}}
\expandafter\def\csname PY@tok@nf\endcsname{\def\PY@tc##1{\textcolor[rgb]{0.00,0.00,1.00}{##1}}}
\expandafter\def\csname PY@tok@nc\endcsname{\let\PY@bf=\textbf\def\PY@tc##1{\textcolor[rgb]{0.00,0.00,1.00}{##1}}}
\expandafter\def\csname PY@tok@nn\endcsname{\let\PY@bf=\textbf\def\PY@tc##1{\textcolor[rgb]{0.00,0.00,1.00}{##1}}}
\expandafter\def\csname PY@tok@ne\endcsname{\let\PY@bf=\textbf\def\PY@tc##1{\textcolor[rgb]{0.82,0.25,0.23}{##1}}}
\expandafter\def\csname PY@tok@nv\endcsname{\def\PY@tc##1{\textcolor[rgb]{0.10,0.09,0.49}{##1}}}
\expandafter\def\csname PY@tok@no\endcsname{\def\PY@tc##1{\textcolor[rgb]{0.53,0.00,0.00}{##1}}}
\expandafter\def\csname PY@tok@nl\endcsname{\def\PY@tc##1{\textcolor[rgb]{0.63,0.63,0.00}{##1}}}
\expandafter\def\csname PY@tok@ni\endcsname{\let\PY@bf=\textbf\def\PY@tc##1{\textcolor[rgb]{0.60,0.60,0.60}{##1}}}
\expandafter\def\csname PY@tok@na\endcsname{\def\PY@tc##1{\textcolor[rgb]{0.49,0.56,0.16}{##1}}}
\expandafter\def\csname PY@tok@nt\endcsname{\let\PY@bf=\textbf\def\PY@tc##1{\textcolor[rgb]{0.00,0.50,0.00}{##1}}}
\expandafter\def\csname PY@tok@nd\endcsname{\def\PY@tc##1{\textcolor[rgb]{0.67,0.13,1.00}{##1}}}
\expandafter\def\csname PY@tok@s\endcsname{\def\PY@tc##1{\textcolor[rgb]{0.73,0.13,0.13}{##1}}}
\expandafter\def\csname PY@tok@sd\endcsname{\let\PY@it=\textit\def\PY@tc##1{\textcolor[rgb]{0.73,0.13,0.13}{##1}}}
\expandafter\def\csname PY@tok@si\endcsname{\let\PY@bf=\textbf\def\PY@tc##1{\textcolor[rgb]{0.73,0.40,0.53}{##1}}}
\expandafter\def\csname PY@tok@se\endcsname{\let\PY@bf=\textbf\def\PY@tc##1{\textcolor[rgb]{0.73,0.40,0.13}{##1}}}
\expandafter\def\csname PY@tok@sr\endcsname{\def\PY@tc##1{\textcolor[rgb]{0.73,0.40,0.53}{##1}}}
\expandafter\def\csname PY@tok@ss\endcsname{\def\PY@tc##1{\textcolor[rgb]{0.10,0.09,0.49}{##1}}}
\expandafter\def\csname PY@tok@sx\endcsname{\def\PY@tc##1{\textcolor[rgb]{0.00,0.50,0.00}{##1}}}
\expandafter\def\csname PY@tok@m\endcsname{\def\PY@tc##1{\textcolor[rgb]{0.40,0.40,0.40}{##1}}}
\expandafter\def\csname PY@tok@gh\endcsname{\let\PY@bf=\textbf\def\PY@tc##1{\textcolor[rgb]{0.00,0.00,0.50}{##1}}}
\expandafter\def\csname PY@tok@gu\endcsname{\let\PY@bf=\textbf\def\PY@tc##1{\textcolor[rgb]{0.50,0.00,0.50}{##1}}}
\expandafter\def\csname PY@tok@gd\endcsname{\def\PY@tc##1{\textcolor[rgb]{0.63,0.00,0.00}{##1}}}
\expandafter\def\csname PY@tok@gi\endcsname{\def\PY@tc##1{\textcolor[rgb]{0.00,0.63,0.00}{##1}}}
\expandafter\def\csname PY@tok@gr\endcsname{\def\PY@tc##1{\textcolor[rgb]{1.00,0.00,0.00}{##1}}}
\expandafter\def\csname PY@tok@ge\endcsname{\let\PY@it=\textit}
\expandafter\def\csname PY@tok@gs\endcsname{\let\PY@bf=\textbf}
\expandafter\def\csname PY@tok@gp\endcsname{\let\PY@bf=\textbf\def\PY@tc##1{\textcolor[rgb]{0.00,0.00,0.50}{##1}}}
\expandafter\def\csname PY@tok@go\endcsname{\def\PY@tc##1{\textcolor[rgb]{0.53,0.53,0.53}{##1}}}
\expandafter\def\csname PY@tok@gt\endcsname{\def\PY@tc##1{\textcolor[rgb]{0.00,0.27,0.87}{##1}}}
\expandafter\def\csname PY@tok@err\endcsname{\def\PY@bc##1{\setlength{\fboxsep}{0pt}\fcolorbox[rgb]{1.00,0.00,0.00}{1,1,1}{\strut ##1}}}
\expandafter\def\csname PY@tok@kc\endcsname{\let\PY@bf=\textbf\def\PY@tc##1{\textcolor[rgb]{0.00,0.50,0.00}{##1}}}
\expandafter\def\csname PY@tok@kd\endcsname{\let\PY@bf=\textbf\def\PY@tc##1{\textcolor[rgb]{0.00,0.50,0.00}{##1}}}
\expandafter\def\csname PY@tok@kn\endcsname{\let\PY@bf=\textbf\def\PY@tc##1{\textcolor[rgb]{0.00,0.50,0.00}{##1}}}
\expandafter\def\csname PY@tok@kr\endcsname{\let\PY@bf=\textbf\def\PY@tc##1{\textcolor[rgb]{0.00,0.50,0.00}{##1}}}
\expandafter\def\csname PY@tok@bp\endcsname{\def\PY@tc##1{\textcolor[rgb]{0.00,0.50,0.00}{##1}}}
\expandafter\def\csname PY@tok@fm\endcsname{\def\PY@tc##1{\textcolor[rgb]{0.00,0.00,1.00}{##1}}}
\expandafter\def\csname PY@tok@vc\endcsname{\def\PY@tc##1{\textcolor[rgb]{0.10,0.09,0.49}{##1}}}
\expandafter\def\csname PY@tok@vg\endcsname{\def\PY@tc##1{\textcolor[rgb]{0.10,0.09,0.49}{##1}}}
\expandafter\def\csname PY@tok@vi\endcsname{\def\PY@tc##1{\textcolor[rgb]{0.10,0.09,0.49}{##1}}}
\expandafter\def\csname PY@tok@vm\endcsname{\def\PY@tc##1{\textcolor[rgb]{0.10,0.09,0.49}{##1}}}
\expandafter\def\csname PY@tok@sa\endcsname{\def\PY@tc##1{\textcolor[rgb]{0.73,0.13,0.13}{##1}}}
\expandafter\def\csname PY@tok@sb\endcsname{\def\PY@tc##1{\textcolor[rgb]{0.73,0.13,0.13}{##1}}}
\expandafter\def\csname PY@tok@sc\endcsname{\def\PY@tc##1{\textcolor[rgb]{0.73,0.13,0.13}{##1}}}
\expandafter\def\csname PY@tok@dl\endcsname{\def\PY@tc##1{\textcolor[rgb]{0.73,0.13,0.13}{##1}}}
\expandafter\def\csname PY@tok@s2\endcsname{\def\PY@tc##1{\textcolor[rgb]{0.73,0.13,0.13}{##1}}}
\expandafter\def\csname PY@tok@sh\endcsname{\def\PY@tc##1{\textcolor[rgb]{0.73,0.13,0.13}{##1}}}
\expandafter\def\csname PY@tok@s1\endcsname{\def\PY@tc##1{\textcolor[rgb]{0.73,0.13,0.13}{##1}}}
\expandafter\def\csname PY@tok@mb\endcsname{\def\PY@tc##1{\textcolor[rgb]{0.40,0.40,0.40}{##1}}}
\expandafter\def\csname PY@tok@mf\endcsname{\def\PY@tc##1{\textcolor[rgb]{0.40,0.40,0.40}{##1}}}
\expandafter\def\csname PY@tok@mh\endcsname{\def\PY@tc##1{\textcolor[rgb]{0.40,0.40,0.40}{##1}}}
\expandafter\def\csname PY@tok@mi\endcsname{\def\PY@tc##1{\textcolor[rgb]{0.40,0.40,0.40}{##1}}}
\expandafter\def\csname PY@tok@il\endcsname{\def\PY@tc##1{\textcolor[rgb]{0.40,0.40,0.40}{##1}}}
\expandafter\def\csname PY@tok@mo\endcsname{\def\PY@tc##1{\textcolor[rgb]{0.40,0.40,0.40}{##1}}}
\expandafter\def\csname PY@tok@ch\endcsname{\let\PY@it=\textit\def\PY@tc##1{\textcolor[rgb]{0.25,0.50,0.50}{##1}}}
\expandafter\def\csname PY@tok@cm\endcsname{\let\PY@it=\textit\def\PY@tc##1{\textcolor[rgb]{0.25,0.50,0.50}{##1}}}
\expandafter\def\csname PY@tok@cpf\endcsname{\let\PY@it=\textit\def\PY@tc##1{\textcolor[rgb]{0.25,0.50,0.50}{##1}}}
\expandafter\def\csname PY@tok@c1\endcsname{\let\PY@it=\textit\def\PY@tc##1{\textcolor[rgb]{0.25,0.50,0.50}{##1}}}
\expandafter\def\csname PY@tok@cs\endcsname{\let\PY@it=\textit\def\PY@tc##1{\textcolor[rgb]{0.25,0.50,0.50}{##1}}}

\def\PYZbs{\char`\\}
\def\PYZus{\char`\_}
\def\PYZob{\char`\{}
\def\PYZcb{\char`\}}
\def\PYZca{\char`\^}
\def\PYZam{\char`\&}
\def\PYZlt{\char`\<}
\def\PYZgt{\char`\>}
\def\PYZsh{\char`\#}
\def\PYZpc{\char`\%}
\def\PYZdl{\char`\$}
\def\PYZhy{\char`\-}
\def\PYZsq{\char`\'}
\def\PYZdq{\char`\"}
\def\PYZti{\char`\~}
% for compatibility with earlier versions
\def\PYZat{@}
\def\PYZlb{[}
\def\PYZrb{]}
\makeatother


    % For linebreaks inside Verbatim environment from package fancyvrb. 
    \makeatletter
        \newbox\Wrappedcontinuationbox 
        \newbox\Wrappedvisiblespacebox 
        \newcommand*\Wrappedvisiblespace {\textcolor{red}{\textvisiblespace}} 
        \newcommand*\Wrappedcontinuationsymbol {\textcolor{red}{\llap{\tiny$\m@th\hookrightarrow$}}} 
        \newcommand*\Wrappedcontinuationindent {3ex } 
        \newcommand*\Wrappedafterbreak {\kern\Wrappedcontinuationindent\copy\Wrappedcontinuationbox} 
        % Take advantage of the already applied Pygments mark-up to insert 
        % potential linebreaks for TeX processing. 
        %        {, <, #, %, $, ' and ": go to next line. 
        %        _, }, ^, &, >, - and ~: stay at end of broken line. 
        % Use of \textquotesingle for straight quote. 
        \newcommand*\Wrappedbreaksatspecials {% 
            \def\PYGZus{\discretionary{\char`\_}{\Wrappedafterbreak}{\char`\_}}% 
            \def\PYGZob{\discretionary{}{\Wrappedafterbreak\char`\{}{\char`\{}}% 
            \def\PYGZcb{\discretionary{\char`\}}{\Wrappedafterbreak}{\char`\}}}% 
            \def\PYGZca{\discretionary{\char`\^}{\Wrappedafterbreak}{\char`\^}}% 
            \def\PYGZam{\discretionary{\char`\&}{\Wrappedafterbreak}{\char`\&}}% 
            \def\PYGZlt{\discretionary{}{\Wrappedafterbreak\char`\<}{\char`\<}}% 
            \def\PYGZgt{\discretionary{\char`\>}{\Wrappedafterbreak}{\char`\>}}% 
            \def\PYGZsh{\discretionary{}{\Wrappedafterbreak\char`\#}{\char`\#}}% 
            \def\PYGZpc{\discretionary{}{\Wrappedafterbreak\char`\%}{\char`\%}}% 
            \def\PYGZdl{\discretionary{}{\Wrappedafterbreak\char`\$}{\char`\$}}% 
            \def\PYGZhy{\discretionary{\char`\-}{\Wrappedafterbreak}{\char`\-}}% 
            \def\PYGZsq{\discretionary{}{\Wrappedafterbreak\textquotesingle}{\textquotesingle}}% 
            \def\PYGZdq{\discretionary{}{\Wrappedafterbreak\char`\"}{\char`\"}}% 
            \def\PYGZti{\discretionary{\char`\~}{\Wrappedafterbreak}{\char`\~}}% 
        } 
        % Some characters . , ; ? ! / are not pygmentized. 
        % This macro makes them "active" and they will insert potential linebreaks 
        \newcommand*\Wrappedbreaksatpunct {% 
            \lccode`\~`\.\lowercase{\def~}{\discretionary{\hbox{\char`\.}}{\Wrappedafterbreak}{\hbox{\char`\.}}}% 
            \lccode`\~`\,\lowercase{\def~}{\discretionary{\hbox{\char`\,}}{\Wrappedafterbreak}{\hbox{\char`\,}}}% 
            \lccode`\~`\;\lowercase{\def~}{\discretionary{\hbox{\char`\;}}{\Wrappedafterbreak}{\hbox{\char`\;}}}% 
            \lccode`\~`\:\lowercase{\def~}{\discretionary{\hbox{\char`\:}}{\Wrappedafterbreak}{\hbox{\char`\:}}}% 
            \lccode`\~`\?\lowercase{\def~}{\discretionary{\hbox{\char`\?}}{\Wrappedafterbreak}{\hbox{\char`\?}}}% 
            \lccode`\~`\!\lowercase{\def~}{\discretionary{\hbox{\char`\!}}{\Wrappedafterbreak}{\hbox{\char`\!}}}% 
            \lccode`\~`\/\lowercase{\def~}{\discretionary{\hbox{\char`\/}}{\Wrappedafterbreak}{\hbox{\char`\/}}}% 
            \catcode`\.\active
            \catcode`\,\active 
            \catcode`\;\active
            \catcode`\:\active
            \catcode`\?\active
            \catcode`\!\active
            \catcode`\/\active 
            \lccode`\~`\~ 	
        }
    \makeatother

    \let\OriginalVerbatim=\Verbatim
    \makeatletter
    \renewcommand{\Verbatim}[1][1]{%
        %\parskip\z@skip
        \sbox\Wrappedcontinuationbox {\Wrappedcontinuationsymbol}%
        \sbox\Wrappedvisiblespacebox {\FV@SetupFont\Wrappedvisiblespace}%
        \def\FancyVerbFormatLine ##1{\hsize\linewidth
            \vtop{\raggedright\hyphenpenalty\z@\exhyphenpenalty\z@
                \doublehyphendemerits\z@\finalhyphendemerits\z@
                \strut ##1\strut}%
        }%
        % If the linebreak is at a space, the latter will be displayed as visible
        % space at end of first line, and a continuation symbol starts next line.
        % Stretch/shrink are however usually zero for typewriter font.
        \def\FV@Space {%
            \nobreak\hskip\z@ plus\fontdimen3\font minus\fontdimen4\font
            \discretionary{\copy\Wrappedvisiblespacebox}{\Wrappedafterbreak}
            {\kern\fontdimen2\font}%
        }%
        
        % Allow breaks at special characters using \PYG... macros.
        \Wrappedbreaksatspecials
        % Breaks at punctuation characters . , ; ? ! and / need catcode=\active 	
        \OriginalVerbatim[#1,codes*=\Wrappedbreaksatpunct]%
    }
    \makeatother

    % Exact colors from NB
    \definecolor{incolor}{HTML}{303F9F}
    \definecolor{outcolor}{HTML}{D84315}
    \definecolor{cellborder}{HTML}{CFCFCF}
    \definecolor{cellbackground}{HTML}{F7F7F7}
    
    % prompt
    \makeatletter
    \newcommand{\boxspacing}{\kern\kvtcb@left@rule\kern\kvtcb@boxsep}
    \makeatother
    \newcommand{\prompt}[4]{
        \ttfamily\llap{{\color{#2}[#3]:\hspace{3pt}#4}}\vspace{-\baselineskip}
    }
    

    
    % Prevent overflowing lines due to hard-to-break entities
    \sloppy 
    % Setup hyperref package
    \hypersetup{
      breaklinks=true,  % so long urls are correctly broken across lines
      colorlinks=true,
      urlcolor=urlcolor,
      linkcolor=linkcolor,
      citecolor=citecolor,
      }
    % Slightly bigger margins than the latex defaults
    

    
    \usepackage[spanish]{babel}
    \usepackage{multicol}
    
    
    \geometry{tmargin=0.5in, bmargin=0.7in, lmargin=0.45in, rmargin=2.5in}
    \graphicspath{assets/}
    
    

\begin{document}
    
    \maketitle
    \tableofcontents
    
    

\newpage
    \hypertarget{tarea-5}{%
\section{Tarea 5}\label{tarea-5}}

\emph{Tarea 5} de \emph{Benjamín Rivera} para el curso de
\textbf{Métodos Numéricos} impartido por \emph{Joaquín Peña Acevedo}.
Fecha limite de entrega \textbf{4 de Octubre de 2020}.

    \hypertarget{como-ejecutar}{%
\subsubsection{Como ejecutar}\label{como-ejecutar}}

    \hypertarget{requerimientos}{%
\subparagraph{Requerimientos}\label{requerimientos}}

Este programa se ejecuto en mi computadora con la version de
\textbf{Python 3.8.2} y con estos
\href{https://github.com/BenchHPZ/UG-Compu/blob/master/MN/requerimientos.txt}{requerimientos}

\hypertarget{jupyter}{%
\paragraph{Jupyter}\label{jupyter}}

En caso de tener acceso a un \emph{servidor jupyter} ,con los
requerimientos antes mencionados, unicamente basta con ejecutar todas
las celdas de este \emph{notebook}. Probablemente no todas las celdas de
\emph{markdown} produzcan el mismo resultado por las
\href{jupyter-contrib-nbextensions.readthedocs.io}{\emph{Nbextensions}}.

\hypertarget{consola}{%
\paragraph{Consola}\label{consola}}

Habrá archivos e instrucciones para poder ejecutar cada uno de los
ejercicios desde la consola.

\hypertarget{si-todo-sale-mal}{%
\paragraph{Si todo sale mal}\label{si-todo-sale-mal}}

En caso de que todo salga mal, tratare de dejar una copia disponible en
\textbf{GoogleColab} que se pueda ejecutar con la versión de
\textbf{Python} de \emph{GoogleColab}

    \begin{tcolorbox}[breakable, size=fbox, boxrule=1pt, pad at break*=1mm,colback=cellbackground, colframe=cellborder]
\prompt{In}{incolor}{138}{\boxspacing}
\begin{Verbatim}[commandchars=\\\{\}]
\PY{n}{usage} \PY{o}{=} \PY{l+s+s2}{\PYZdq{}\PYZdq{}\PYZdq{}}
\PY{l+s+s2}{Programa correspondiente a la Tarea 5 de Metodos Numericos. }
\PY{l+s+s2}{Este programa espera leer los archivos de tipo npy}

\PY{l+s+s2}{Alumno: Benjamin Rivera}

\PY{l+s+s2}{Usage:}
\PY{l+s+s2}{  Tarea5.py ejercicio1 \PYZlt{}matA\PYZgt{} \PYZlt{}vecB\PYZgt{} \PYZlt{}N\PYZgt{}[\PYZhy{}\PYZhy{}path=\PYZlt{}path\PYZgt{}]}
\PY{l+s+s2}{  Tarea5.py \PYZhy{}h | \PYZhy{}\PYZhy{}help}

\PY{l+s+s2}{Options:}
\PY{l+s+s2}{  \PYZhy{}h \PYZhy{}\PYZhy{}help       Show this screen.}
\PY{l+s+s2}{  \PYZhy{}v \PYZhy{}\PYZhy{}version    Show version.}
\PY{l+s+s2}{  \PYZhy{}\PYZhy{}path=\PYZlt{}path\PYZgt{}   Directorio para buscar archivos [default: data/].}
\PY{l+s+s2}{\PYZdq{}\PYZdq{}\PYZdq{}}
\PY{k+kn}{import} \PY{n+nn}{sys}
\PY{k+kn}{import} \PY{n+nn}{scipy}
\PY{k+kn}{import} \PY{n+nn}{numpy} \PY{k}{as} \PY{n+nn}{np}
\PY{k+kn}{import} \PY{n+nn}{matplotlib}\PY{n+nn}{.}\PY{n+nn}{pyplot} \PY{k}{as} \PY{n+nn}{plt}
\PY{k+kn}{from} \PY{n+nn}{scipy}\PY{n+nn}{.}\PY{n+nn}{linalg} \PY{k+kn}{import} \PY{n}{solve\PYZus{}triangular}

\PY{k}{if} \PY{n+nv+vm}{\PYZus{}\PYZus{}name\PYZus{}\PYZus{}} \PY{o}{==} \PY{l+s+s2}{\PYZdq{}}\PY{l+s+s2}{\PYZus{}\PYZus{}main\PYZus{}\PYZus{}}\PY{l+s+s2}{\PYZdq{}}\PY{p}{:}
    \PY{k+kn}{import} \PY{n+nn}{doctest}
    \PY{k+kn}{from} \PY{n+nn}{docopt} \PY{k+kn}{import} \PY{n}{docopt}
    \PY{n}{doctest}\PY{o}{.}\PY{n}{testmod}\PY{p}{(}\PY{p}{)}
    \PY{n}{args} \PY{o}{=} \PY{n}{docopt}\PY{p}{(}\PY{n}{usage}\PY{p}{,} \PY{n}{version}\PY{o}{=}\PY{l+s+s1}{\PYZsq{}}\PY{l+s+s1}{Tarea4, prb}\PY{l+s+s1}{\PYZsq{}}\PY{p}{)}
    

    \PY{k}{if} \PY{n}{args}\PY{p}{[}\PY{l+s+s1}{\PYZsq{}}\PY{l+s+s1}{ejercicio3}\PY{l+s+s1}{\PYZsq{}}\PY{p}{]}\PY{p}{:}
        \PY{n}{Ejercicio3}\PY{p}{(}\PY{n}{args}\PY{p}{[}\PY{l+s+s1}{\PYZsq{}}\PY{l+s+s1}{\PYZlt{}matA\PYZgt{}}\PY{l+s+s1}{\PYZsq{}}\PY{p}{]}\PY{p}{,} \PY{n}{args}\PY{p}{[}\PY{l+s+s1}{\PYZsq{}}\PY{l+s+s1}{\PYZlt{}vecB\PYZgt{}}\PY{l+s+s1}{\PYZsq{}}\PY{p}{]}\PY{p}{,} \PY{n}{args}\PY{p}{[}\PY{l+s+s1}{\PYZsq{}}\PY{l+s+s1}{\PYZlt{}N\PYZgt{}}\PY{l+s+s1}{\PYZsq{}}\PY{p}{]}\PY{p}{,} \PY{n}{args}\PY{p}{[}\PY{l+s+s1}{\PYZsq{}}\PY{l+s+s1}{\PYZhy{}\PYZhy{}path}\PY{l+s+s1}{\PYZsq{}}\PY{p}{]}\PY{p}{)}
\end{Verbatim}
\end{tcolorbox}




















\newpage
    \hypertarget{ejercicio-1}{%
\subsection{Ejercicio 1}\label{ejercicio-1}}

    Considere la matriz \begin{equation*}
    A = \begin{pmatrix}
            a^2 & a & a/2 & 1 \\
            a & -9 & 1 & 0 \\
            a/2 & 1 & 10 & 0 \\
            1 & 0 & 0 & a
        \end{pmatrix}
\end{equation*}

Da un rango de valores para \(a\) de manera que garantice la
convergencia del método de Jacobi.

    \hypertarget{resp}{%
\subsubsection{Respuesta}\label{resp}}

    Por las notas del curso (ppt clase 9, diapositiva 8-41), sabemos que el
método de Jacobi converge cuando la matriz \(A\) es
\textbf{estrictamente diagonal dominante}. Y esto es cierto cuando

\begin{equation*}
    \forall i \in [1,\dots,n] , |a_{i,i}| > \sum_{j=1, j\neq i}^{n} |a_{i,j}|
\end{equation*}

si extendemos estas desigualdades para la matriz \(A\) nos queda que

\begin{eqnarray}
    sol &=&
    \begin{cases}
        |a^2| &> |a| + |a/2| + |1| \\
        |-9|  &> |a| + |1| + |0| \\
        |10|  &> |a/2| + |1| + |0| \\
        |a|  &> |1| + |0| + |0| 
    \end{cases} \\
    && \text{despues de simplificar queda que} \\
    &=& \begin{cases}
        a^2 &> |a| + |a/2| + 1 \\
        8   &> |a| \\
        9  &> |a/2| \\
        |a| &> 1 
    \end{cases} \\
    && \\
    &=& \begin{cases}
        a^2 &> |a| + |a/2| + 1 \\
        64   &> a^2 \\
        4*91  &> a^2 \\
        a^2 &> 1 
    \end{cases} \label{eq: red}\\
    && \text{realcionamos \ref{eq: red}.4 con \ref{eq: red}.3 y \ref{eq: red}.2} \\
    &=& \begin{cases}
        a^2 &> |a| + |a/2| + 1 \\
        8^2   &> a^2 >  1\\
        4*9^2  &> a^2 > 1\\
    \end{cases} \label{eq: cuadrado}\\
    && \text{de esto solo nos importa \ref{eq: cuadrado}.1 y \ref{eq: cuadrado}.2} \\
    &=& \begin{cases}
        a^2 &> 3|a|/2 + 1 \\
        8   &> a >  1\\
    \end{cases} \label{eq: final}\\
\end{eqnarray}

Podemos calcular los intervalos de soluci'on de \ref{eq: final}. Estos
quedan

\begin{equation}
    \begin{cases}
        a^2 > 3|a|/2 + 1 \Rightarrow& (-\infty, -2) \cup(2, \infty)  \\
        8   > a >  1     \Rightarrow& (1, 8)
    \end{cases}
    \label{eq: casi sol}
\end{equation}

Y la solucion que buscamos es la interseccion de \ref{eq: casi sol}. De
manera que, para que la matriz \(A\) converja con el metodo de Jacobi,
se necesita que \(x \in (2,8)\).




















\newpage
    \hypertarget{ejercicio-2}{%
\subsection{Ejercicio 2}\label{ejercicio-2}}

    Sea \(A \in \mathbb R^{n\times n}\) una matriz tridiagonal y que las
tres diagonales de interes se almacenan en un arreglo
\(B_{n \times 3}\).

Escribe las expresiones para calcular las actualizaciones de las
componentes del vector
\(x^{i+1} = \left( x^{i+1}_0, \dots, x^{i+1}_{n-1} \right)\) de acuerdo
con \textit{Gauss-Seidel}. Especificamente escribir la expresion para
actualizar \(x^{i+1}_0, x^{i+1}_i\) para \(i = 1,2,\dots, n-2\); ademas
de \(x^{i+1}_{n-1}\) usando los coeficientes \(a_{i,j}\) de \(A\) y
\(b_{ij}\) de \(B\).

    \hypertarget{respuesta}{%
\subsubsection{Respuesta}\label{respuesta}}

    \par

Sea \(A\) una matriz tridiagonal con elementos \(a_{i,j}\), \(B\) el
arreglo descrito anteriormente con elementos \(b'_{i,j}\), \(b\) el
vector de terminos independientes con elementos \(b_i\) y \(x\) el
vector solucion con \(x_i^{(t)}\) su elemento \(i\) de la iteraci'on
\(t\).

\par

Se da que en el m'etodo de \textit{Gauss-Seidel} original tenemos que
los componentes \(x^{(t+1)}\) se calculan siguiendo
\verb|forwardSubstitution|

\begin{equation}
    x_{i}^{(t+1)} = \frac{1}{a_{i,i}}\left( b_i - \sum_{j=0}^{i-1} a_{j,j}x_{j}^{(t+1)} - \sum_{j=i+1}^{n-1} a_{i,j}x_j^{(t)}\right)
    \label{eq: GS original}
\end{equation}

\noindent pero como en este ejercicio estamos trabajando con una matriz
tridiagonal, lo que implica que solo habr'a elementos distitnos de cero
en las tres diagonales de interes; entonces podemos reescribir la
ecuaci'on\textasciitilde{}\ref{eq: GS original}, lo que queda como

\begin{eqnarray*}
    x_{i}^{(t+1)} &=& \frac{1}{a_{i,i}}\left( b_i - a_{i, j-1}x_{i-1}^{t+1} - a_{i, j+1}x_{i+1}^{t}\right)
    \label{eq: GS tridiagonal} \\
     &=& \frac{1}{b'_{1,i}}\left( b_i - b'_{i,0}x_{i-1}^{t+1} - b'_{i,2}x_{i+1}^{t}\right)
    \qquad\text{Usando el arreglo B} \label{eq: GS tridiagonal con B}
\end{eqnarray*}

\noindent esto se puede usar \(\forall i = 0,1,\dots,n-1,n\) sobre el
arreglo \(B\).

\par

Espec'ificamente podemos definir al elemento \(x_0^{t+1}\) como
\begin{eqnarray*}
    x_0^{t+1} &=& \frac{1}{a_{0,0}} \left( b_0 - a_{0, 1}x_1^{t}  \right) \\
        &=& \frac{1}{b_{0,1}} \left( b_0 - b'_{0, 2}x_1^{t}  \right)
\end{eqnarray*}

\noindent y para el elemento \(x_{n-1}^{i+1}\) queda que
\begin{eqnarray*}
    x_{n-1}^{i+1} &=& \frac{1}{a_{n-1,n-1}} \left( b_{n-1} - a_{n-1, n-2}x_{n-2}^{t+1}  \right) \\
        &=& \frac{1}{a_{n-1,1}} \left( b_{n-1} - b'_{n-1, 0}x_{n-2}^{t+1}  \right)
\end{eqnarray*}




















\newpage
    \hypertarget{ejercicio-3}{%
\subsection{Ejercicio 3}\label{ejercicio-3}}

    Programa el metodo de \textit{Gauss-Seidel} para resolver sistemas
tridiagonales.

    \begin{tcolorbox}[breakable, size=fbox, boxrule=1pt, pad at break*=1mm,colback=cellbackground, colframe=cellborder]
\prompt{In}{incolor}{65}{\boxspacing}
\begin{Verbatim}[commandchars=\\\{\}]
\PY{c+c1}{\PYZsh{} Extras}

\PY{k}{def} \PY{n+nf}{data4mFile}\PY{p}{(}\PY{n}{n\PYZus{}file}\PY{p}{,}\PY{o}{/}\PY{p}{,}\PY{n}{path}\PY{o}{=}\PY{l+s+s1}{\PYZsq{}}\PY{l+s+s1}{datos/npy/}\PY{l+s+s1}{\PYZsq{}}\PY{p}{,} \PY{n}{ext}\PY{o}{=}\PY{l+s+s1}{\PYZsq{}}\PY{l+s+s1}{npy}\PY{l+s+s1}{\PYZsq{}}\PY{p}{,} \PY{n}{dtype}\PY{o}{=}\PY{n}{np}\PY{o}{.}\PY{n}{float64}\PY{p}{)}\PY{p}{:}
    \PY{l+s+sd}{\PYZdq{}\PYZdq{}\PYZdq{} Cargar matrices y vectores de memoria}
\PY{l+s+sd}{    }
\PY{l+s+sd}{    Funcion para cargar los archivos en memoria. El nombre del}
\PY{l+s+sd}{    archivo no espera path, ni la extension, solo el nombre. Por }
\PY{l+s+sd}{    default trata de leer los archivos .npy, pero numpy soporta }
\PY{l+s+sd}{    leer de otros formatos.}
\PY{l+s+sd}{    }
\PY{l+s+sd}{        Input:}
\PY{l+s+sd}{            n\PYZus{}file := nombre del archivo sin extension}
\PY{l+s+sd}{            path := directorio para buscar el archivo}
\PY{l+s+sd}{            ext := extension del archivo a buscar (sin punto)}
\PY{l+s+sd}{            dtype := tipo de dato para guardar los valores}
\PY{l+s+sd}{        Output:}
\PY{l+s+sd}{            Regresa el una instancia np.matrix con los datos}
\PY{l+s+sd}{                obtenidos del archivo cargado.}
\PY{l+s+sd}{    \PYZdq{}\PYZdq{}\PYZdq{}}



\PY{k}{def} \PY{n+nf}{show1D}\PY{p}{(}\PY{n}{vec}\PY{p}{,}\PY{o}{/}\PY{p}{,} \PY{n}{max\PYZus{}sz}\PY{o}{=}\PY{l+m+mi}{8}\PY{p}{,} \PY{n}{show}\PY{o}{=}\PY{k+kc}{True}\PY{p}{)}\PY{p}{:}
    \PY{l+s+sd}{\PYZdq{}\PYZdq{}\PYZdq{} Implementacion para pprint vector 1D.}
\PY{l+s+sd}{    }
\PY{l+s+sd}{    Funcion para generar string para poder imporimir un}
\PY{l+s+sd}{    vector de manera reducida, dando un maximo de elementos }
\PY{l+s+sd}{    a imprimir. Lo puede imprimir directamente si se quiere}
\PY{l+s+sd}{    }
\PY{l+s+sd}{    Input:}
\PY{l+s+sd}{        vec := vector de informacion a imprimir.}
\PY{l+s+sd}{        [opcionales]}
\PY{l+s+sd}{        max\PYZus{}sz := Maximo de elementos a imprimir.}
\PY{l+s+sd}{        show := Imprimir vector}
\PY{l+s+sd}{        }
\PY{l+s+sd}{    \PYZus{}Doctest:}
\PY{l+s+sd}{        \PYZgt{}\PYZgt{}\PYZgt{} show1D([1,2,3,4], show=False)}
\PY{l+s+sd}{        \PYZsq{}1, 2, 3, 4\PYZsq{}}
\PY{l+s+sd}{        }
\PY{l+s+sd}{        \PYZgt{}\PYZgt{}\PYZgt{} show1D([1,2,3,4,5,6,7,8,9], show=False)}
\PY{l+s+sd}{        \PYZsq{}1, 2, 3, 4, ..., 6, 7, 8, 9\PYZsq{}}
\PY{l+s+sd}{    \PYZdq{}\PYZdq{}\PYZdq{}}
     
     
     
\end{Verbatim}
\end{tcolorbox}



    \begin{tcolorbox}[breakable, size=fbox, boxrule=1pt, pad at break*=1mm,colback=cellbackground, colframe=cellborder]
\prompt{In}{incolor}{143}{\boxspacing}
\begin{Verbatim}[commandchars=\\\{\}]
\PY{c+c1}{\PYZsh{} Parte 1}

\PY{k}{def} \PY{n+nf}{diagonalesRelevantes}\PY{p}{(}\PY{n}{A}\PY{p}{,} \PY{n}{dtype}\PY{o}{=}\PY{n}{np}\PY{o}{.}\PY{n}{float64}\PY{p}{)}\PY{p}{:}
    \PY{l+s+sd}{\PYZdq{}\PYZdq{}\PYZdq{} Funcion que otiene las diagonales relevantes de A.}
\PY{l+s+sd}{    }
\PY{l+s+sd}{    Esta funcion, con A una matriz tridiagonal cuadrada(n) extrae\PYZus{}}
\PY{l+s+sd}{    ra las diagonales relevantes y las pondra en un arreglo B de }
\PY{l+s+sd}{    3xn, donde la columna 0 correspondera a la diagonal \PYZhy{}1, la col\PYZus{}}
\PY{l+s+sd}{    umna 1 a la diagonal de A y la columna 2 a la diagonal +1 de}
\PY{l+s+sd}{    la matriz A.}
\PY{l+s+sd}{    Se espera, y corrobora, que A sea instancia de np.matrix para}
\PY{l+s+sd}{    usar su metodos}
\PY{l+s+sd}{    }
\PY{l+s+sd}{    Input:}
\PY{l+s+sd}{        A := Matriz tridiagonal cuadrada instancia de np.matrix}
\PY{l+s+sd}{    Output:}
\PY{l+s+sd}{        B := Arreglo de valores relevantes de A}
\PY{l+s+sd}{    }
\PY{l+s+sd}{    \PYZdq{}\PYZdq{}\PYZdq{}}
    \PY{k}{if} \PY{n+nb}{isinstance}\PY{p}{(}\PY{n}{A}\PY{p}{,} \PY{p}{(}\PY{n}{np}\PY{o}{.}\PY{n}{matrix}\PY{p}{)}\PY{p}{)}\PY{p}{:}
        \PY{k}{return} \PY{n}{B}
    \PY{k}{else}\PY{p}{:}
        \PY{k}{raise} \PY{n+ne}{Exception}\PY{p}{(}\PY{l+s+s2}{\PYZdq{}}\PY{l+s+s2}{A no es instancia de np.matrix}\PY{l+s+s2}{\PYZdq{}}\PY{p}{)} 
\end{Verbatim}
\end{tcolorbox}


    \begin{tcolorbox}[breakable, size=fbox, boxrule=1pt, pad at break*=1mm,colback=cellbackground, colframe=cellborder]
\prompt{In}{incolor}{190}{\boxspacing}
\begin{Verbatim}[commandchars=\\\{\}]
\PY{c+c1}{\PYZsh{} Parte 2}

\PY{k}{def} \PY{n+nf}{error\PYZus{}GS}\PY{p}{(}\PY{n}{B}\PY{p}{,} \PY{n}{xt}\PY{p}{,} \PY{n}{b}\PY{p}{,}\PY{o}{/}\PY{p}{,} \PY{n}{dtype}\PY{o}{=}\PY{n}{np}\PY{o}{.}\PY{n}{float64}\PY{p}{)}\PY{p}{:}
    \PY{l+s+sd}{\PYZdq{}\PYZdq{}\PYZdq{} Funcion para calular el error || Ax\PYZca{}t \PYZhy{} b|| desde B \PYZdq{}\PYZdq{}\PYZdq{}}    
    \PY{n}{n} \PY{o}{=} \PY{n+nb}{len}\PY{p}{(}\PY{n}{xt}\PY{p}{)} \PY{c+c1}{\PYZsh{} esperamos que las dimensiones coincidan}
    \PY{n}{vec} \PY{o}{=} \PY{n}{np}\PY{o}{.}\PY{n}{asmatrix}\PY{p}{(}\PY{n}{np}\PY{o}{.}\PY{n}{zeros}\PY{p}{(}\PY{p}{(}\PY{n}{n}\PY{p}{,}\PY{l+m+mi}{1}\PY{p}{)}\PY{p}{)}\PY{p}{,}
                      \PY{n}{dtype}\PY{o}{=}\PY{n}{dtype}\PY{p}{)}
    \PY{c+c1}{\PYZsh{} En vec generaremos Ax\PYZca{}t}
    
    \PY{n}{vec}\PY{p}{[}\PY{l+m+mi}{0}\PY{p}{,}\PY{l+m+mi}{0}\PY{p}{]} \PY{o}{=} \PY{n}{B}\PY{p}{[}\PY{l+m+mi}{0}\PY{p}{,}\PY{l+m+mi}{1}\PY{p}{]}\PY{o}{*}\PY{n}{xt}\PY{p}{[}\PY{l+m+mi}{0}\PY{p}{,}\PY{l+m+mi}{0}\PY{p}{]} \PY{o}{+} \PY{n}{B}\PY{p}{[}\PY{l+m+mi}{0}\PY{p}{,}\PY{l+m+mi}{2}\PY{p}{]}\PY{o}{*}\PY{n}{xt}\PY{p}{[}\PY{l+m+mi}{1}\PY{p}{,}\PY{l+m+mi}{0}\PY{p}{]} 
    
    \PY{c+c1}{\PYZsh{} Calculamos hasta el penultimo}
    \PY{k}{for} \PY{n}{i} \PY{o+ow}{in} \PY{n+nb}{range}\PY{p}{(}\PY{l+m+mi}{1}\PY{p}{,} \PY{n}{n}\PY{o}{\PYZhy{}}\PY{l+m+mi}{1}\PY{p}{)}\PY{p}{:}
        \PY{n}{vec}\PY{p}{[}\PY{n}{i}\PY{p}{,}\PY{l+m+mi}{0}\PY{p}{]} \PY{o}{=} \PY{n}{B}\PY{p}{[}\PY{n}{i}\PY{p}{,}\PY{l+m+mi}{0}\PY{p}{]}\PY{o}{*}\PY{n}{xt}\PY{p}{[}\PY{n}{i}\PY{o}{\PYZhy{}}\PY{l+m+mi}{1}\PY{p}{,}\PY{l+m+mi}{0}\PY{p}{]} \PY{o}{+} \PY{n}{B}\PY{p}{[}\PY{n}{i}\PY{p}{,}\PY{l+m+mi}{1}\PY{p}{]}\PY{o}{*}\PY{n}{xt}\PY{p}{[}\PY{n}{i}\PY{p}{,}\PY{l+m+mi}{0}\PY{p}{]} \PY{o}{+} \PY{n}{B}\PY{p}{[}\PY{n}{i}\PY{p}{,}\PY{l+m+mi}{2}\PY{p}{]}\PY{o}{*}\PY{n}{xt}\PY{p}{[}\PY{n}{i}\PY{o}{+}\PY{l+m+mi}{1}\PY{p}{,}\PY{l+m+mi}{0}\PY{p}{]}
    
    \PY{n}{n} \PY{o}{=} \PY{n}{n}\PY{o}{\PYZhy{}}\PY{l+m+mi}{1} \PY{c+c1}{\PYZsh{} Calculamos el ultimo}
    \PY{n}{vec}\PY{p}{[}\PY{n}{n}\PY{p}{,}\PY{l+m+mi}{0}\PY{p}{]} \PY{o}{=} \PY{n}{B}\PY{p}{[}\PY{n}{n}\PY{p}{,}\PY{l+m+mi}{0}\PY{p}{]}\PY{o}{*}\PY{n}{xt}\PY{p}{[}\PY{n}{n}\PY{o}{\PYZhy{}}\PY{l+m+mi}{1}\PY{p}{,}\PY{l+m+mi}{0}\PY{p}{]} \PY{o}{+} \PY{n}{B}\PY{p}{[}\PY{n}{n}\PY{p}{,}\PY{l+m+mi}{1}\PY{p}{]}\PY{o}{*}\PY{n}{xt}\PY{p}{[}\PY{n}{n}\PY{p}{,}\PY{l+m+mi}{0}\PY{p}{]}
    
    \PY{k}{return} \PY{n}{np}\PY{o}{.}\PY{n}{linalg}\PY{o}{.}\PY{n}{norm}\PY{p}{(}\PY{n}{vec} \PY{o}{\PYZhy{}} \PY{n}{b}\PY{p}{)}

\end{Verbatim}
\end{tcolorbox}


    \begin{tcolorbox}[breakable, size=fbox, boxrule=1pt, pad at break*=1mm,colback=cellbackground, colframe=cellborder]
\prompt{In}{incolor}{190}{\boxspacing}
\begin{Verbatim}[commandchars=\\\{\}]

\PY{k}{def} \PY{n+nf}{GaussSeidel\PYZus{}tridiagonal}\PY{p}{(}\PY{n}{B}\PY{p}{,} \PY{n}{xt}\PY{p}{,} \PY{n}{b}\PY{p}{,} \PY{n}{N}\PY{p}{,}\PY{o}{/}\PY{p}{,} \PY{n}{t}\PY{o}{=}\PY{k+kc}{None}\PY{p}{,} \PY{n}{dtype}\PY{o}{=}\PY{n}{np}\PY{o}{.}\PY{n}{float64}\PY{p}{)}\PY{p}{:}
    \PY{l+s+sd}{\PYZdq{}\PYZdq{}\PYZdq{} Implementacion de GaussSeidel para matrices tridiagonales.}
\PY{l+s+sd}{    }
\PY{l+s+sd}{    Esta funcion trata de resolver un sistema de ecuaciones Ax = b }
\PY{l+s+sd}{    con A una matriz (nxn) cuadrada tridiagonal y estas diagonales }
\PY{l+s+sd}{    almacenadas en el arreglo B (3xn). }
\PY{l+s+sd}{    }
\PY{l+s+sd}{    Respecto a la tolerancia t del metodo, en caso de ser None, se}
\PY{l+s+sd}{    tomara el epsilon para el tipo de dato dtype que se le pase a }
\PY{l+s+sd}{    la funcion (calculado por numpy)}
\PY{l+s+sd}{    }
\PY{l+s+sd}{    Input:}
\PY{l+s+sd}{        B := arreglo (3xn) de la diagonales relecantes para el metodo}
\PY{l+s+sd}{        x0 := vector (nx1) inicial de aproximacion de respuestas}
\PY{l+s+sd}{        b :=  vector (nx1) de terminos independientes}
\PY{l+s+sd}{        N := maximo numero de iteraciones del metodo}
\PY{l+s+sd}{        }
\PY{l+s+sd}{        t := Tolerancia del metodo (default: None)}
\PY{l+s+sd}{        dtype := Tipop de dato para trabajar con el metodo}
\PY{l+s+sd}{    }
\PY{l+s+sd}{    Output:}
\PY{l+s+sd}{        x, n, e}
\PY{l+s+sd}{        x := vector respuesta en la iteracion en que se detenga}
\PY{l+s+sd}{        n := iteracion en la que se detuvo el metodo}
\PY{l+s+sd}{        e := error al momento de detenerse}
\PY{l+s+sd}{    \PYZdq{}\PYZdq{}\PYZdq{}}
    \PY{c+c1}{\PYZsh{} Inicializacion}
    \PY{k}{if} \PY{n}{t} \PY{o}{==} \PY{k+kc}{None}\PY{p}{:} \PY{n}{t} \PY{o}{=} \PY{n}{np}\PY{o}{.}\PY{n}{finfo}\PY{p}{(}\PY{n}{dtype}\PY{p}{)}\PY{o}{.}\PY{n}{eps}   \PY{c+c1}{\PYZsh{} Correcion tolerancia}
    \PY{n}{sz} \PY{o}{=} \PY{n+nb}{len}\PY{p}{(}\PY{n}{b}\PY{p}{)}
    \PY{n}{e} \PY{o}{=} \PY{n+nb}{float}\PY{p}{(}\PY{l+s+s1}{\PYZsq{}}\PY{l+s+s1}{inf}\PY{l+s+s1}{\PYZsq{}}\PY{p}{)}                  \PY{c+c1}{\PYZsh{} Error inicial es infinito}
    \PY{n}{n} \PY{o}{=} \PY{l+m+mi}{0}                                      \PY{c+c1}{\PYZsh{} Iteracion inicial}
    
    \PY{k}{while} \PY{n}{n} \PY{o}{\PYZlt{}} \PY{n}{N}\PY{p}{:}
        
        \PY{c+c1}{\PYZsh{} Primer elemento de iteracion}
        \PY{n}{i} \PY{o}{=} \PY{l+m+mi}{0}
        \PY{n}{xt}\PY{p}{[}\PY{n}{i}\PY{p}{,}\PY{l+m+mi}{0}\PY{p}{]} \PY{o}{=} \PY{p}{(}\PY{n}{b}\PY{p}{[}\PY{n}{i}\PY{p}{,}\PY{l+m+mi}{0}\PY{p}{]} \PY{o}{\PYZhy{}} \PY{n}{B}\PY{p}{[}\PY{n}{i}\PY{p}{,}\PY{l+m+mi}{2}\PY{p}{]}\PY{o}{*}\PY{n}{xt}\PY{p}{[}\PY{n}{i}\PY{o}{+}\PY{l+m+mi}{1}\PY{p}{,}\PY{l+m+mi}{0}\PY{p}{]}\PY{p}{)}\PY{o}{/}\PY{n}{B}\PY{p}{[}\PY{n}{i}\PY{p}{,}\PY{l+m+mi}{1}\PY{p}{]}
        \PY{c+c1}{\PYZsh{} Iteracion del metodo}
        \PY{k}{for} \PY{n}{i} \PY{o+ow}{in} \PY{n+nb}{range}\PY{p}{(}\PY{l+m+mi}{1}\PY{p}{,} \PY{n}{sz}\PY{o}{\PYZhy{}}\PY{l+m+mi}{1}\PY{p}{)}\PY{p}{:}
            \PY{n}{xt}\PY{p}{[}\PY{n}{i}\PY{p}{,}\PY{l+m+mi}{0}\PY{p}{]} \PY{o}{=} \PY{p}{(}\PY{n}{b}\PY{p}{[}\PY{n}{i}\PY{p}{,}\PY{l+m+mi}{0}\PY{p}{]} \PY{o}{\PYZhy{}} \PY{n}{B}\PY{p}{[}\PY{n}{i}\PY{p}{,}\PY{l+m+mi}{0}\PY{p}{]}\PY{o}{*}\PY{n}{xt}\PY{p}{[}\PY{n}{i}\PY{o}{\PYZhy{}}\PY{l+m+mi}{1}\PY{p}{,}\PY{l+m+mi}{0}\PY{p}{]} \PY{o}{\PYZhy{}} \PY{n}{B}\PY{p}{[}\PY{n}{i}\PY{p}{,}\PY{l+m+mi}{2}\PY{p}{]}\PY{o}{*}\PY{n}{xt}\PY{p}{[}\PY{n}{i}\PY{o}{+}\PY{l+m+mi}{1}\PY{p}{,}\PY{l+m+mi}{0}\PY{p}{]}\PY{p}{)}\PY{o}{/}\PY{n}{B}\PY{p}{[}\PY{n}{i}\PY{p}{,}\PY{l+m+mi}{1}\PY{p}{]}
        \PY{c+c1}{\PYZsh{} Ultimo elemento de iteracion}
        \PY{n}{i} \PY{o}{=} \PY{n}{sz}\PY{o}{\PYZhy{}}\PY{l+m+mi}{1}
        \PY{n}{xt}\PY{p}{[}\PY{n}{i}\PY{p}{,}\PY{l+m+mi}{0}\PY{p}{]} \PY{o}{=} \PY{p}{(}\PY{n}{b}\PY{p}{[}\PY{n}{i}\PY{p}{,}\PY{l+m+mi}{0}\PY{p}{]} \PY{o}{\PYZhy{}} \PY{n}{B}\PY{p}{[}\PY{n}{i}\PY{p}{,}\PY{l+m+mi}{0}\PY{p}{]}\PY{o}{*}\PY{n}{xt}\PY{p}{[}\PY{n}{i}\PY{o}{\PYZhy{}}\PY{l+m+mi}{1}\PY{p}{,}\PY{l+m+mi}{0}\PY{p}{]}\PY{p}{)}\PY{o}{/}\PY{n}{B}\PY{p}{[}\PY{n}{i}\PY{p}{,}\PY{l+m+mi}{1}\PY{p}{]}
        
        \PY{c+c1}{\PYZsh{} avance bucle verificacion tolerancia}
        \PY{n}{e} \PY{o}{=} \PY{n}{error\PYZus{}GS}\PY{p}{(}\PY{n}{B}\PY{p}{,} \PY{n}{xt}\PY{p}{,} \PY{n}{b}\PY{p}{,} \PY{n}{dtype}\PY{o}{=}\PY{n}{dtype}\PY{p}{)}
        \PY{k}{if} \PY{n}{e} \PY{o}{\PYZlt{}} \PY{n}{t}\PY{p}{:} 
            \PY{k}{break}
        \PY{n}{n} \PY{o}{+}\PY{o}{=} \PY{l+m+mi}{1}
            
    \PY{k}{return} \PY{n}{xt}\PY{p}{,} \PY{n}{n}\PY{p}{,} \PY{n}{e}
    
\end{Verbatim}
\end{tcolorbox}

    \begin{tcolorbox}[breakable, size=fbox, boxrule=1pt, pad at break*=1mm,colback=cellbackground, colframe=cellborder]
\prompt{In}{incolor}{191}{\boxspacing}
\begin{Verbatim}[commandchars=\\\{\}]
\PY{c+c1}{\PYZsh{} Parte 3}

\PY{k}{def} \PY{n+nf}{Ejercicio3}\PY{p}{(}\PY{n}{mat}\PY{p}{,} \PY{n}{vecb}\PY{p}{,} \PY{n}{N}\PY{p}{,}\PY{o}{/}\PY{p}{,} \PY{n}{path}\PY{o}{=}\PY{l+s+s1}{\PYZsq{}}\PY{l+s+s1}{datos/npy/}\PY{l+s+s1}{\PYZsq{}}\PY{p}{,} \PY{n}{show}\PY{o}{=}\PY{k+kc}{True}\PY{p}{)}\PY{p}{:}
    \PY{l+s+sd}{\PYZdq{}\PYZdq{}\PYZdq{} Funcion para ejecutar la parte 3 de la tarea }
\PY{l+s+sd}{    }
\PY{l+s+sd}{    Esta funcion usara las funciones diagonalesRelevantes, error\PYZus{}GS,}
\PY{l+s+sd}{    GaussSeidel\PYZus{}tridiagonal, data4mFile y show1D para tratar de }
\PY{l+s+sd}{    resolver un sistema Ax = b mediante la variante del metodo de}
\PY{l+s+sd}{    Gauss\PYZhy{}Seidel para matrices tridiagonales cuadradas}
\PY{l+s+sd}{    }
\PY{l+s+sd}{    Input:}
\PY{l+s+sd}{        mat := nombre del archivo que contiene una matriz}
\PY{l+s+sd}{            tridiagonal}
\PY{l+s+sd}{        vecb := nombre del archivo con el vector de terminos}
\PY{l+s+sd}{            independientes}
\PY{l+s+sd}{        N := numero maximo de iteraciones para el metodo}
\PY{l+s+sd}{        }
\PY{l+s+sd}{        path := directorio para buscar los archivos}
\PY{l+s+sd}{        show := Indica si se desea imprimir los detalles}
\PY{l+s+sd}{    \PYZdq{}\PYZdq{}\PYZdq{}}
    \PY{n}{dtype} \PY{o}{=} \PY{n}{np}\PY{o}{.}\PY{n}{float64}
    \PY{n}{t} \PY{o}{=} \PY{p}{(}\PY{n}{np}\PY{o}{.}\PY{n}{finfo}\PY{p}{(}\PY{n}{dtype}\PY{p}{)}\PY{o}{.}\PY{n}{eps}\PY{p}{)}\PY{o}{*}\PY{o}{*}\PY{p}{(}\PY{l+m+mi}{1}\PY{o}{/}\PY{l+m+mi}{2}\PY{p}{)}
    
    \PY{n}{A} \PY{o}{=} \PY{n}{data4mFile}\PY{p}{(}\PY{n}{mat}\PY{p}{,}  \PY{n}{dtype}\PY{o}{=}\PY{n}{dtype}\PY{p}{)}
    \PY{n}{b} \PY{o}{=} \PY{n}{data4mFile}\PY{p}{(}\PY{n}{vecb}\PY{p}{,} \PY{n}{dtype}\PY{o}{=}\PY{n}{dtype}\PY{p}{)}\PY{o}{.}\PY{n}{transpose}\PY{p}{(}\PY{p}{)}
    
    \PY{n}{x0} \PY{o}{=} \PY{n}{np}\PY{o}{.}\PY{n}{zeros}\PY{p}{(}\PY{n}{b}\PY{o}{.}\PY{n}{shape}\PY{p}{,} \PY{n}{dtype}\PY{o}{=}\PY{n}{dtype}\PY{p}{)}
    
    \PY{c+c1}{\PYZsh{} Suponemos que A si es tridiagonal}
    \PY{n}{B} \PY{o}{=} \PY{n}{diagonalesRelevantes}\PY{p}{(}\PY{n}{A}\PY{p}{,} \PY{n}{dtype}\PY{o}{=}\PY{n}{dtype}\PY{p}{)}
    
    \PY{n}{xt}\PY{p}{,} \PY{n}{n}\PY{p}{,} \PY{n}{e} \PY{o}{=} \PY{n}{GaussSeidel\PYZus{}tridiagonal}\PY{p}{(}\PY{n}{B}\PY{p}{,} \PY{n}{x0}\PY{p}{,} \PY{n}{b}\PY{p}{,} \PY{n}{N}\PY{p}{,} \PY{n}{t}\PY{o}{=}\PY{n}{t}\PY{p}{,} \PY{n}{dtype}\PY{o}{=}\PY{n}{dtype}\PY{p}{)}
    \PY{n}{conv} \PY{o}{=} \PY{k+kc}{True} \PY{k}{if} \PY{n}{e} \PY{o}{\PYZlt{}} \PY{n}{t} \PY{k}{else} \PY{k+kc}{False}
    
    \PY{k}{if} \PY{n}{show}\PY{p}{:}
        \PY{c+c1}{\PYZsh{} Segunda solucion}
        \PY{n}{x} \PY{o}{=} \PY{n}{np}\PY{o}{.}\PY{n}{linalg}\PY{o}{.}\PY{n}{solve}\PY{p}{(}\PY{n}{A}\PY{p}{,} \PY{n}{b}\PY{p}{)}
        \PY{c+c1}{\PYZsh{} Print}
        \PY{n}{\PYZus{}\PYZus{}} \PY{o}{=} \PY{l+s+sa}{f}\PY{l+s+s1}{\PYZsq{}}\PY{l+s+s1}{Matriz de }\PY{l+s+s1}{\PYZdq{}}\PY{l+s+si}{\PYZob{}}\PY{n}{mat}\PY{l+s+si}{\PYZcb{}}\PY{l+s+s1}{\PYZdq{}}\PY{l+s+s1}{ con el vector de }\PY{l+s+s1}{\PYZdq{}}\PY{l+s+si}{\PYZob{}}\PY{n}{vecb}\PY{l+s+si}{\PYZcb{}}\PY{l+s+s1}{\PYZdq{}}\PY{l+s+s1}{\PYZsq{}}
        \PY{n}{\PYZus{}\PYZus{}} \PY{o}{+}\PY{o}{=} \PY{l+s+sa}{f}\PY{l+s+s1}{\PYZsq{}}\PY{l+s+se}{\PYZbs{}n}\PY{l+s+se}{\PYZbs{}t}\PY{l+s+s1}{Iteraciones: }\PY{l+s+si}{\PYZob{}}\PY{n}{n}\PY{l+s+si}{\PYZcb{}}\PY{l+s+s1}{\PYZsq{}}
        \PY{n}{\PYZus{}\PYZus{}} \PY{o}{+}\PY{o}{=} \PY{l+s+sa}{f}\PY{l+s+s1}{\PYZsq{}}\PY{l+s+se}{\PYZbs{}n}\PY{l+s+se}{\PYZbs{}t}\PY{l+s+s1}{Error: }\PY{l+s+si}{\PYZob{}}\PY{n}{e}\PY{l+s+si}{\PYZcb{}}\PY{l+s+s1}{\PYZsq{}}
        \PY{n}{\PYZus{}\PYZus{}} \PY{o}{+}\PY{o}{=} \PY{l+s+sa}{f}\PY{l+s+s1}{\PYZsq{}}\PY{l+s+se}{\PYZbs{}n}\PY{l+s+se}{\PYZbs{}t}\PY{l+s+s1}{Sol: }\PY{l+s+si}{\PYZob{}}\PY{n}{show1D}\PY{p}{(}\PY{n}{xt}\PY{p}{,}\PY{n}{show}\PY{o}{=}\PY{k+kc}{False}\PY{p}{)}\PY{l+s+si}{\PYZcb{}}\PY{l+s+se}{\PYZbs{}n}\PY{l+s+s1}{\PYZsq{}}
        \PY{n}{\PYZus{}\PYZus{}} \PY{o}{+}\PY{o}{=} \PY{p}{(}\PY{l+s+s1}{\PYZsq{}}\PY{l+s+s1}{El metodo converge}\PY{l+s+s1}{\PYZsq{}} \PY{k}{if} \PY{n}{e} \PY{o}{\PYZlt{}} \PY{n}{t} \PY{k}{else} \PY{l+s+s1}{\PYZsq{}}\PY{l+s+s1}{El metodo no converge}\PY{l+s+s1}{\PYZsq{}}\PY{p}{)}
        \PY{n}{\PYZus{}\PYZus{}} \PY{o}{+}\PY{o}{=} \PY{l+s+sa}{f}\PY{l+s+s1}{\PYZsq{}}\PY{l+s+se}{\PYZbs{}n}\PY{l+s+s1}{La diferencia entre soluciones es }\PY{l+s+si}{\PYZob{}}\PY{n}{np}\PY{o}{.}\PY{n}{linalg}\PY{o}{.}\PY{n}{norm}\PY{p}{(}\PY{n}{x} \PY{o}{\PYZhy{}} \PY{n}{xt}\PY{p}{)}\PY{l+s+si}{\PYZcb{}}\PY{l+s+s1}{\PYZsq{}}
        \PY{n+nb}{print}\PY{p}{(}\PY{n}{\PYZus{}\PYZus{}}\PY{p}{)}
    
    \PY{k}{return} \PY{n}{e}\PY{p}{,} \PY{n}{n}\PY{p}{,} \PY{n}{conv}
\end{Verbatim}
\end{tcolorbox}

    \begin{tcolorbox}[breakable, size=fbox, boxrule=1pt, pad at break*=1mm,colback=cellbackground, colframe=cellborder]
\prompt{In}{incolor}{240}{\boxspacing}
\begin{Verbatim}[commandchars=\\\{\}]
\PY{c+c1}{\PYZsh{} Parte 4}

\PY{k}{if} \PY{n}{NOTEBOOK}\PY{p}{:}
    \PY{n}{sizes} \PY{o}{=} \PY{p}{[}\PY{l+s+s1}{\PYZsq{}}\PY{l+s+s1}{6}\PY{l+s+s1}{\PYZsq{}}\PY{p}{,} \PY{l+s+s1}{\PYZsq{}}\PY{l+s+s1}{20}\PY{l+s+s1}{\PYZsq{}}\PY{p}{,} \PY{l+s+s1}{\PYZsq{}}\PY{l+s+s1}{500}\PY{l+s+s1}{\PYZsq{}}\PY{p}{]}
    \PY{n}{data} \PY{o}{=} \PY{p}{\PYZob{}}\PY{p}{\PYZcb{}}
    \PY{k}{for} \PY{n}{sz} \PY{o+ow}{in} \PY{n}{sizes}\PY{p}{:}
        \PY{n}{data}\PY{p}{[}\PY{n}{sz}\PY{p}{]} \PY{o}{=} \PY{p}{[}\PY{p}{[}\PY{p}{]}\PY{p}{,}\PY{p}{[}\PY{p}{]}\PY{p}{,}\PY{p}{[}\PY{p}{]}\PY{p}{]}
        
    \PY{n}{itr} \PY{o}{=} \PY{p}{[}\PY{l+m+mi}{0}\PY{p}{,} \PY{l+m+mi}{5}\PY{p}{,} \PY{l+m+mi}{10}\PY{p}{,} \PY{l+m+mi}{15}\PY{p}{,} \PY{l+m+mi}{20}\PY{p}{,} \PY{l+m+mi}{25}\PY{p}{,} \PY{l+m+mi}{35}\PY{p}{,} \PY{l+m+mi}{50}\PY{p}{]}

    \PY{k}{for} \PY{n}{sz} \PY{o+ow}{in} \PY{n}{sizes}\PY{p}{:}
        \PY{k}{for} \PY{n}{N} \PY{o+ow}{in} \PY{n}{itr}\PY{p}{:}
            \PY{n}{e}\PY{p}{,} \PY{n}{n}\PY{p}{,} \PY{n}{conv} \PY{o}{=} \PY{n}{Ejercicio3}\PY{p}{(}\PY{l+s+s1}{\PYZsq{}}\PY{l+s+s1}{matrizA}\PY{l+s+s1}{\PYZsq{}}\PY{o}{+}\PY{n}{sz}\PY{p}{,} \PY{l+s+s1}{\PYZsq{}}\PY{l+s+s1}{vecb}\PY{l+s+s1}{\PYZsq{}}\PY{o}{+}\PY{n}{sz}\PY{p}{,} \PY{n}{N}\PY{p}{,} \PY{n}{show}\PY{o}{=}\PY{k+kc}{True}\PY{p}{)}
            \PY{n}{data}\PY{p}{[}\PY{n}{sz}\PY{p}{]}\PY{p}{[}\PY{l+m+mi}{0}\PY{p}{]}\PY{o}{.}\PY{n}{append}\PY{p}{(}\PY{n}{e}\PY{p}{)}
            \PY{n}{data}\PY{p}{[}\PY{n}{sz}\PY{p}{]}\PY{p}{[}\PY{l+m+mi}{1}\PY{p}{]}\PY{o}{.}\PY{n}{append}\PY{p}{(}\PY{n}{n}\PY{p}{)}
            \PY{n}{data}\PY{p}{[}\PY{n}{sz}\PY{p}{]}\PY{p}{[}\PY{l+m+mi}{2}\PY{p}{]}\PY{o}{.}\PY{n}{append}\PY{p}{(}\PY{n}{conv}\PY{p}{)}
\end{Verbatim}
\end{tcolorbox}

    \begin{Verbatim}[commandchars=\\\{\}]
Matriz de "matrizA6" con el vector de "vecb6"
        Iteraciones: 0
        Error: inf
        Sol: 0.0, 0.0, 0.0, 0.0, 0.0, 0.0
El metodo no converge
La diferencia entre soluciones es 2.449489742783178
Matriz de "matrizA6" con el vector de "vecb6"
        Iteraciones: 5
        Error: 0.00022626705409276946
        Sol: 0.9999715782516367, 1.000046095638286, 0.9999528204288628,
1.000020636310501, 0.9999923716262922, 1.0000011924468593
El metodo no converge
La diferencia entre soluciones es 7.51264727492206e-05
Matriz de "matrizA6" con el vector de "vecb6"
        Iteraciones: 9
        Error: 1.8517803127045294e-09
        Sol: 1.0000000003609595, 0.9999999996391488, 1.0000000003090268,
0.9999999998701277, 1.0000000000473834, 0.9999999999925931
El metodo converge
La diferencia entre soluciones es 6.125110654833705e-10
Matriz de "matrizA6" con el vector de "vecb6"
        Iteraciones: 9
        Error: 1.8517803127045294e-09
        Sol: 1.0000000003609595, 0.9999999996391488, 1.0000000003090268,
0.9999999998701277, 1.0000000000473834, 0.9999999999925931
El metodo converge
La diferencia entre soluciones es 6.125110654833705e-10
Matriz de "matrizA6" con el vector de "vecb6"
        Iteraciones: 9
        Error: 1.8517803127045294e-09
        Sol: 1.0000000003609595, 0.9999999996391488, 1.0000000003090268,
0.9999999998701277, 1.0000000000473834, 0.9999999999925931
El metodo converge
La diferencia entre soluciones es 6.125110654833705e-10
Matriz de "matrizA6" con el vector de "vecb6"
        Iteraciones: 9
        Error: 1.8517803127045294e-09
        Sol: 1.0000000003609595, 0.9999999996391488, 1.0000000003090268,
0.9999999998701277, 1.0000000000473834, 0.9999999999925931
El metodo converge
La diferencia entre soluciones es 6.125110654833705e-10
Matriz de "matrizA6" con el vector de "vecb6"
        Iteraciones: 9
        Error: 1.8517803127045294e-09
        Sol: 1.0000000003609595, 0.9999999996391488, 1.0000000003090268,
0.9999999998701277, 1.0000000000473834, 0.9999999999925931
El metodo converge
La diferencia entre soluciones es 6.125110654833705e-10
Matriz de "matrizA6" con el vector de "vecb6"
        Iteraciones: 9
        Error: 1.8517803127045294e-09
        Sol: 1.0000000003609595, 0.9999999996391488, 1.0000000003090268,
0.9999999998701277, 1.0000000000473834, 0.9999999999925931
El metodo converge
La diferencia entre soluciones es 6.125110654833705e-10
Matriz de "matrizA20" con el vector de "vecb20"
        Iteraciones: 0
        Error: inf
        Sol: 0.0, 0.0, 0.0, 0.0, {\ldots}, 0.0, 0.0, 0.0, 0.0
El metodo no converge
La diferencia entre soluciones es 44.72135954999579
Matriz de "matrizA20" con el vector de "vecb20"
        Iteraciones: 5
        Error: 5.2135922638126534e+17
        Sol: 10510715.21822592, 1872849342.4323115, 335161007203.7978,
30692230637880.035, {\ldots}, 4.099992817956197e+16, 4.332425697334634e+17,
3.365085604102188e+17, 6.572227164204795e+17
El metodo no converge
La diferencia entre soluciones es 8.573144572254433e+17
Matriz de "matrizA20" con el vector de "vecb20"
        Iteraciones: 10
        Error: 2.1453562634774947e+30
        Sol: -4.325493091370833e+19, -7.706464998382885e+21,
-1.3791289121479942e+24, -1.2629315580338113e+26, {\ldots}, -1.687103028015392e+29,
-1.782760425966517e+30, -1.3847085054535812e+30, -2.704424173623813e+30
El metodo no converge
La diferencia entre soluciones es 3.527786386918865e+30
Matriz de "matrizA20" con el vector de "vecb20"
        Iteraciones: 15
        Error: 8.827765445581566e+42
        Sol: 1.7798646821865153e+32, 3.171075432413957e+34,
5.674874034154363e+36, 5.196742264243895e+38, {\ldots}, 6.942133605708313e+41,
7.335747052314032e+42, 5.69783308475235e+42, 1.1128232022110846e+43
El metodo no converge
La diferencia entre soluciones es 1.451622338717728e+43
Matriz de "matrizA20" con el vector de "vecb20"
        Iteraciones: 20
        Error: 3.6324709362165934e+55
        Sol: -7.323831572435901e+44, -1.3048420255161528e+47,
-2.3351113169951127e+49, -2.1383684641646873e+51, {\ldots}, -2.8565664452140027e+54,
-3.0185314876376157e+55, -2.3445585643800182e+55, -4.5790726589847025e+55
El metodo no converge
La diferencia entre soluciones es 5.97317179331505e+55
Matriz de "matrizA20" con el vector de "vecb20"
        Iteraciones: 25
        Error: 1.4946981978392363e+68
        Sol: 3.0136284762679543e+57, 5.3691965008129655e+59,
9.608574269562953e+61, 8.799011873257374e+63, {\ldots}, 1.1754270832835685e+67,
1.242072862774836e+68, 9.647448038654873e+67, 1.8842082349289435e+68
El metodo no converge
La diferencia entre soluciones es 2.4578556226939094e+68
Matriz de "matrizA20" con el vector de "vecb20"
        Iteraciones: 35
        Error: 2.5307907379234587e+93
        Sol: 5.102610711852639e+82, 9.091007665622408e+84,
1.6269030631878133e+87, 1.4898297050140288e+89, {\ldots}, 1.990207775575571e+92,
2.1030509714140467e+93, 1.6334850858936836e+93, 3.190300728381666e+93
El metodo no converge
La diferencia entre soluciones es 4.1615881079264364e+93
Matriz de "matrizA20" con el vector de "vecb20"
        Iteraciones: 50
        Error: 1.7632355142337908e+131
        Sol: -3.5550566420321894e+120, -6.3338257628323356e+122,
-1.1334849682523423e+125, -1.0379841393747967e+127, {\ldots},
-1.3866041858041658e+130, -1.4652235388231731e+131, -1.1380707508761582e+131,
-2.2227248824152343e+131
El metodo no converge
La diferencia entre soluciones es 2.8994336977579156e+131
Matriz de "matrizA500" con el vector de "vecb500"
        Iteraciones: 0
        Error: inf
        Sol: 0.0, 0.0, 0.0, 0.0, {\ldots}, 0.0, 0.0, 0.0, 0.0
El metodo no converge
La diferencia entre soluciones es 2236.06797749979
Matriz de "matrizA500" con el vector de "vecb500"
        Iteraciones: 5
        Error: 14.176187474228898
        Sol: -99.63958099545093, 99.47440006331433, -99.59286032243953,
99.70598909314587, {\ldots}, -99.91668866583471, 99.9348818313726,
-99.96581153196503, 99.99173629461029
El metodo no converge
La diferencia entre soluciones es 5.874263087048714
Matriz de "matrizA500" con el vector de "vecb500"
        Iteraciones: 10
        Error: 0.038973335267423345
        Sol: -99.99917521831776, 99.99893028154601, -99.99925001852736,
99.99948437479507, {\ldots}, -99.99998299574862, 99.99998753009527,
-99.99999350241926, 99.9999984294677
El metodo no converge
La diferencia entre soluciones es 0.015979939541367193
Matriz de "matrizA500" con el vector de "vecb500"
        Iteraciones: 15
        Error: 0.00010199731641958948
        Sol: -99.99999855757989, 99.99999805927155, -99.99999854223792,
99.99999893159874, {\ldots}, -99.99999999646232, 99.99999999750894,
-99.99999999870863, 99.99999999968786
El metodo no converge
La diferencia entre soluciones es 4.2453048360324766e-05
Matriz de "matrizA500" con el vector de "vecb500"
        Iteraciones: 20
        Error: 2.970999512048245e-07
        Sol: -99.99999999713852, 99.99999999618126, -99.99999999708542,
99.99999999772496, {\ldots}, -99.9999999999992, 99.99999999999949,
-99.99999999999974, 99.99999999999993
El metodo no converge
La diferencia entre soluciones es 1.2607776366756404e-07
Matriz de "matrizA500" con el vector de "vecb500"
        Iteraciones: 22
        Error: 9.286484050814949e-09
        Sol: -99.99999999992775, 99.9999999998994, -99.99999999991924,
99.99999999993344, {\ldots}, -99.99999999999999, 100.0, -100.0, 100.0
El metodo converge
La diferencia entre soluciones es 3.9575573020619025e-09
Matriz de "matrizA500" con el vector de "vecb500"
        Iteraciones: 22
        Error: 9.286484050814949e-09
        Sol: -99.99999999992775, 99.9999999998994, -99.99999999991924,
99.99999999993344, {\ldots}, -99.99999999999999, 100.0, -100.0, 100.0
El metodo converge
La diferencia entre soluciones es 3.9575573020619025e-09
Matriz de "matrizA500" con el vector de "vecb500"
        Iteraciones: 22
        Error: 9.286484050814949e-09
        Sol: -99.99999999992775, 99.9999999998994, -99.99999999991924,
99.99999999993344, {\ldots}, -99.99999999999999, 100.0, -100.0, 100.0
El metodo converge
La diferencia entre soluciones es 3.9575573020619025e-09
    \end{Verbatim}

    \begin{tcolorbox}[breakable, size=fbox, boxrule=1pt, pad at break*=1mm,colback=cellbackground, colframe=cellborder]
\prompt{In}{incolor}{249}{\boxspacing}
\begin{Verbatim}[commandchars=\\\{\}]
\PY{n}{PLOT} \PY{o}{=} \PY{k+kc}{True}
\PY{k}{if} \PY{n}{PLOT}\PY{p}{:}
    \PY{n}{rng} \PY{o}{=} \PY{n}{itr}
    \PY{n}{fig}\PY{p}{,} \PY{n}{ax} \PY{o}{=} \PY{n}{plt}\PY{o}{.}\PY{n}{subplots}\PY{p}{(}\PY{l+m+mi}{2}\PY{p}{,} \PY{l+m+mi}{2}\PY{p}{,} \PY{n}{figsize}\PY{o}{=}\PY{p}{(}\PY{l+m+mi}{10}\PY{p}{,}\PY{l+m+mi}{10}\PY{p}{)}\PY{p}{)}
    \PY{n}{ax}\PY{p}{[}\PY{l+m+mi}{0}\PY{p}{,}\PY{l+m+mi}{1}\PY{p}{]}\PY{o}{.}\PY{n}{axis}\PY{p}{(}\PY{l+s+s1}{\PYZsq{}}\PY{l+s+s1}{off}\PY{l+s+s1}{\PYZsq{}}\PY{p}{)}
    
    \PY{k}{for} \PY{n}{sz} \PY{o+ow}{in} \PY{n}{sizes}\PY{p}{:}
        \PY{k}{if} \PY{n}{sz} \PY{o}{==} \PY{l+s+s1}{\PYZsq{}}\PY{l+s+s1}{6}\PY{l+s+s1}{\PYZsq{}}\PY{p}{:} \PY{n}{i}\PY{p}{,}\PY{n}{j} \PY{o}{=} \PY{l+m+mi}{0}\PY{p}{,}\PY{l+m+mi}{0}
        \PY{k}{elif} \PY{n}{sz} \PY{o}{==} \PY{l+s+s1}{\PYZsq{}}\PY{l+s+s1}{20}\PY{l+s+s1}{\PYZsq{}}\PY{p}{:} \PY{n}{i}\PY{p}{,}\PY{n}{j} \PY{o}{=} \PY{l+m+mi}{1}\PY{p}{,}\PY{l+m+mi}{0}
        \PY{k}{elif} \PY{n}{sz} \PY{o}{==} \PY{l+s+s1}{\PYZsq{}}\PY{l+s+s1}{500}\PY{l+s+s1}{\PYZsq{}}\PY{p}{:} \PY{n}{i}\PY{p}{,}\PY{n}{j} \PY{o}{=} \PY{l+m+mi}{1}\PY{p}{,}\PY{l+m+mi}{1}
        
        \PY{n}{ax}\PY{p}{[}\PY{n}{i}\PY{p}{,}\PY{n}{j}\PY{p}{]}\PY{o}{.}\PY{n}{set\PYZus{}title}\PY{p}{(}\PY{n}{sz}\PY{p}{)}
        \PY{n}{a} \PY{o}{=} \PY{n}{ax}\PY{p}{[}\PY{n}{i}\PY{p}{,}\PY{n}{j}\PY{p}{]}\PY{o}{.}\PY{n}{plot}\PY{p}{(}\PY{n}{rng}\PY{p}{,} \PY{n}{data}\PY{p}{[}\PY{n}{sz}\PY{p}{]}\PY{p}{[}\PY{l+m+mi}{0}\PY{p}{]}\PY{p}{,} \PY{l+s+s1}{\PYZsq{}}\PY{l+s+s1}{\PYZhy{}x}\PY{l+s+s1}{\PYZsq{}}\PY{p}{)}
        \PY{n}{b} \PY{o}{=} \PY{n}{ax}\PY{p}{[}\PY{n}{i}\PY{p}{,}\PY{n}{j}\PY{p}{]}\PY{o}{.}\PY{n}{plot}\PY{p}{(}\PY{n}{rng}\PY{p}{,} \PY{n}{data}\PY{p}{[}\PY{n}{sz}\PY{p}{]}\PY{p}{[}\PY{l+m+mi}{1}\PY{p}{]}\PY{p}{,} \PY{l+s+s1}{\PYZsq{}}\PY{l+s+s1}{\PYZhy{}o}\PY{l+s+s1}{\PYZsq{}}\PY{p}{)}
        \PY{n}{c} \PY{o}{=} \PY{n}{ax}\PY{p}{[}\PY{n}{i}\PY{p}{,}\PY{n}{j}\PY{p}{]}\PY{o}{.}\PY{n}{plot}\PY{p}{(}\PY{n}{rng}\PY{p}{,} \PY{n}{data}\PY{p}{[}\PY{n}{sz}\PY{p}{]}\PY{p}{[}\PY{l+m+mi}{2}\PY{p}{]}\PY{p}{,} \PY{l+s+s1}{\PYZsq{}}\PY{l+s+s1}{*}\PY{l+s+s1}{\PYZsq{}}\PY{p}{)}    
    
    \PY{n}{labels} \PY{o}{=} \PY{p}{[}\PY{l+s+s1}{\PYZsq{}}\PY{l+s+s1}{error}\PY{l+s+s1}{\PYZsq{}}\PY{p}{,} \PY{l+s+s1}{\PYZsq{}}\PY{l+s+s1}{iteraciones}\PY{l+s+s1}{\PYZsq{}}\PY{p}{,} \PY{l+s+s1}{\PYZsq{}}\PY{l+s+s1}{conv}\PY{l+s+s1}{\PYZsq{}}\PY{p}{]}
    \PY{n}{fig}\PY{o}{.}\PY{n}{legend}\PY{p}{(}\PY{p}{[}\PY{n}{e}\PY{p}{,} \PY{n}{i}\PY{p}{,} \PY{n}{c}\PY{p}{]}\PY{p}{,}        \PY{c+c1}{\PYZsh{} The line objects}
           \PY{n}{labels}\PY{o}{=}\PY{n}{labels}\PY{p}{,}        \PY{c+c1}{\PYZsh{} The labels for each line}
           \PY{n}{loc}\PY{o}{=}\PY{l+s+s2}{\PYZdq{}}\PY{l+s+s2}{upper right}\PY{l+s+s2}{\PYZdq{}}\PY{p}{,}    \PY{c+c1}{\PYZsh{} Position of legend}
           \PY{n}{borderaxespad}\PY{o}{=}\PY{l+m+mf}{0.1}\PY{p}{,}    \PY{c+c1}{\PYZsh{} Small spacing around legend box}
           \PY{n}{fontsize}\PY{o}{=}\PY{l+s+s1}{\PYZsq{}}\PY{l+s+s1}{xx\PYZhy{}large}\PY{l+s+s1}{\PYZsq{}}
           \PY{p}{)}
    
    \PY{n}{plt}\PY{o}{.}\PY{n}{show}\PY{p}{(}\PY{p}{)}
\end{Verbatim}
\end{tcolorbox}

    \begin{Verbatim}[commandchars=\\\{\}]
<ipython-input-249-735f8276e7b2>:18: UserWarning: You have mixed positional and
keyword arguments, some input may be discarded.
  fig.legend([e, i, c],        \# The line objects
    \end{Verbatim}

    \begin{center}
    \adjustimage{max size={0.9\linewidth}{0.9\paperheight}}{assets/output_19_1.png}
    \end{center}
    { \hspace*{\fill} \\}
    
    En la figura anterior podemos ver como convergen, o no, el metodo para
los distintos datos proporcionados. Se grafica el error, de manera
directa; las iteraciones con las que termina el metodo; y si converge o
no, donde 0 es no y 1 es si. Todos estos datos se grafican contra el
limite superior de iteraciones que se le pasa al metodo.

    \begin{tcolorbox}[breakable, size=fbox, boxrule=1pt, pad at break*=1mm,colback=cellbackground, colframe=cellborder]
\prompt{In}{incolor}{ }{\boxspacing}
\begin{Verbatim}[commandchars=\\\{\}]

\end{Verbatim}
\end{tcolorbox}


    % Add a bibliography block to the postdoc
    
    
    
\end{document}
